\renewcommand{\abstractname}{Acknowledgements}
\begin{abstract}
  All Masters and PhD theses include a section on acknowledgements, primarily to thank
  all the people who have made the research possible, recognizing the fact that
  research is rarely done in a vacuum, and it is the overall influence of several people
  that comes together to shape both the researcher and the work itself.
  Traditionally, people thank their mentors, collaborators, family,
  friends, and all the other usual suspects. Some acknowledgements also contain fond
  memories of the time spent doing research. And yet, none I that I've seen so far
  talks about the struggles and difficulties endured along the way. Maybe this is by
  choice, only wanting to remember the positives. But I'm going to break tradition
  and provide some background on the conditions under which this work was produced, because
  without such a context, all the thanks are just empty words. Perhaps this is not
  the right place for such a thing, but I don't care, because this is the only place
  I get to share my story. Like most theses, this one will also
  be filed away on some official server, and maybe 4-5 people will
  willingly download it over my lifetime, if I'm lucky. What I hope to achieve
  by sharing this story here is to add to the growing body of material that
  touches upon the dark and painful side of doing research, the emotional distress
  and loneliness it can often bring. I also want to clarify that many people also
  have positive experiences, but that doesn't make mine less worthy of sharing.

  To state it rather bluntly, the only memories I will have of writing this
  thesis is a prolonged struggle with depression.
\end{abstract}
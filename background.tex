% Background: give technical background
\label{ch:background}

In this section, we define the formalize requirements for lazy interpolation-based model checking. This chapter is entirely based on \cite{mcmillan06}. This applies to the standard domain of model checking for programs that don't use heap memory, and we'll extend it to heap-manipulating programs later.

We will use standard first-order logic (FOL) and the notation $\mathcal{L}(\Sigma)$ to denote the set of well-formed formulas (\textit{wffs}) of FOL over a vocabulary $\Sigma$ of non-logical symbols. For a given formula or set of formulas $\phi$ we will use $\mathcal{L}(\phi)$ to denote \textit{wffs} over the vocabulary of $\phi$.

For every non-logical symbol $s$, we presume the existence of a unique symbol $s'$ (that is, $s$ with one prime added). We think of $s$ with $n$ primes added as representing the value of $s$ at $n$ time units in the future. For any formula or term $\phi$, we will use the notation $\phi^{\langle n \rangle}$ to denote the addition of $n$ primes to every symbol in $\phi$ (meaning $\phi$ at $n$ time units in the future). For any set $\Sigma$ of symbols, let $\Sigma'$ denote $\{ s' | s \in \Sigma \}$ and $\Sigma^{\langle n \rangle}$ denote $\{ s^{\langle n \rangle} | s \in \Sigma \}$.

\section{Modeling Programs}
\label{sec:modeling-programs}

We use FOL formulas to characterize programs. To this end, let $S$, the state vocabulary, be a set of individual variables and uninterpreted $n$-ary functional and propositional constants. A \textit{state formula} is a formula in $\mathcal{L}(S)$ (which may also include various interpreted symbols, such as $=$ and $+$). A \textit{transition formula} is a formula in $\mathcal{L}(S \cup S')$.

For our purposes, a \textit{program} is a tuple $(\Lambda, \Delta, l_i, l_f)$, where $\Lambda$ is a finite set of program locations, $\Delta$ is a set of $actions$, $l_i \in \Lambda$ is the initial location and $l_f \in \Lambda$ is the error location. An $action$ is a triple $(l, T, m)$, where $l,m \in \Lambda$ are respectively the entry and exit locations of the action, and $G$ is a transition formula. A $path$ $\pi$ of a program is a sequence of transitions of the form $(l_0, T_0, l_1)(l_1, T_1, l_2) \cdots (l_{n-1}, T_{n-1}, l_n)$. The path is an \textit{error path} when $l_0 = l_1$ and $l_n = l_f$. The unfolding $\mathcal{U}(\pi)$ of path $\pi$ is the sequence of formulas $T_0^{\langle 0 \rangle}, \cdots, T_{n-1}^{\langle n-1 \rangle}$, that is, the sequence of transition formulas $T_0, \cdots, T_{n-1}$, with each $T_i$ shifted $i$ time units into the future.

We will say that path $\pi$ is $feasible$ when $\bigwedge \mathcal{U}(\pi)$ is consistent. We can think of a model of $\bigwedge \mathcal{U}(\pi)$ as a concrete program execution, assigning a value to every program variable at every time $0, \cdots, n-1$. A program is said to be $safe$ when every error path of the program is infeasible. An \textit{inductive invariant} of a program is a map $I : \Lambda \rightarrow \mathcal{L}(S)$, such that $I(l_i) \equiv \true$ and for every action $(l, T, m) \in \Delta$, $I(l) \wedge T$ implies $I(m)'$. A \textit{safety invariant} of a program is an inductive invariant such that $I(l_f) \equiv \false$. Existence of a safety invariant of a program implies that the program is safe.

To simplify the presentation of algorithms, we will assume that every location has at least one outgoing action. This can be made true without affecting program safety by adding self-loops.

\section{Interpolants from Proofs}
\label{sec:interpolants-from-proofs}

Given a pair of formulas $(A,B)$, such that $A \wedge B$ is inconsistent, an \textit{interpolant} for $(A,B)$ is a formula $\hat{A}$ with the following properties:

\begin{itemize}
  \item $A$ implies $\hat{A}$
  \item $\hat{A} \wedge B$ is unsatisfiable, and
  \item $\hat{A} \in \mathcal{L}(A) \cap \mathcal{L}(B)$
\end{itemize}

The Craig Interpolation lemma \cite{craig1957} states that an interpolant always exists for inconsistent formulas in FOL. To handle program paths, this idea can be generalized to sequences of formulas. That is, given a sequence of formulas $\Gamma = A_1, \cdots , A_n$, we say that $\hat{A_0},\cdots, \hat{A_n}$ is an \textit{interpolant} for $\Gamma$ when

\begin{itemize}
  \item $\hat{A_0} = \true$ and $\hat{A_n} = \false$ and,
  \item for all $1 \leq i \leq n, \hat{A_{i-1}} \wedge A_i$ implies $\hat{A_i}$ and
  \item for all $1 \leq i < n, \hat{A_i} \in (\mathcal{L}(A_1 \cdots A_i) \cap \mathcal{L}(A_{i+1}\cdots A_n))$
\end{itemize}

That is, the $i$-th element of the interpolant is a formula over the common vocabulary of the prefix $A_1 \cdots A_i$ and the suffix $A_{i+1} \cdots A_n$, and each interpolant implies the next, with $A_i$. If $\Gamma$ is quantifier-free, we can derive a quantifier-free interpolant for $\Gamma$ from a refutation of $\Gamma$, in certain interpreted theories \cite{mcmillan05}.

\section{Program Unwindings}

We now give a definition of a program unwinding, and describe an algorithm to construct a complete unwinding using interpolants. For two vertices $v$ and $w$ of a tree, we will write $w \sqsubset v$ when $w$ is a proper ancestor of $v$.

% define program unwinding
\begin{defn}
  \label{defn:prog-unwinding}
  An unwinding of a program $\mathcal{A} = (\Lambda, \Delta, l_i, l_f)$ is a quadruple $(V, E, M_v, M_e)$, where $(V, E)$ is a directed tree rooted at $\epsilon$, $M_v : V \rightarrow \Lambda$ is the vertex map, and $M_e : E \rightarrow \Delta$ is the edge map, such that:

  \begin{itemize}
    \item $M_v(\epsilon) = l_i$
    \item for every non-leaf vertex $v \in V$, for every action $(M_v(v), T, m) \in \Delta$, there exists an edge $(v,w) \in E$ such that $M_v(w) = m$ and $M_e(v,w) = T$
  \end{itemize}
\end{defn}

% define labeled program unwinding
\begin{defn}
  \label{defn:labeled-prog-unwinding}
  A labeled unwinding of a program $\mathcal{A} = (\Lambda, \Delta, l_i, l_f)$ is a triple $(U, \psi, \rhd)$, where

  \begin{itemize}
    \item $U = (V, E, M_v, M_e)$ is an unwinding of $\mathcal{A}$
    \item $\psi : V \rightarrow \mathcal{L}(S)$ is called the vertex labeling, and
    \item $\rhd \subseteq V \times V$ is called the covering relation
  \end{itemize}

  A vertex $v \in V$ is said to be covered iff there exists $(w,x) \in \rhd$ such that $w \sqsubseteq v$. The unwinding is said to be safe iff, for all $v \in V$, $M_v(v) = l_f$ implies $\psi(v) \equiv \false$. It is complete iff every leaf $v \in V$ is covered.
\end{defn}

% define well-labeled program unwinding
\begin{defn}
  \label{defn:well-labeled-prog-unwinding}
  A labeled unwinding $(U, \psi, \rhd)$ of a program $\mathcal{A} = (\Lambda, \Delta, l_i, l_f)$, where $U = (V, E, M_v, M_e)$, is said to be well-labeled iff:

  \begin{itemize}
    \item $\psi(\epsilon) \equiv \true$, and
    \item for every edge $(v,w) \in E$, $\psi(v) \wedge M_e(v,w)$ implies $\psi(w)'$, and
    \item for all $(v,w) \in \rhd$, $\psi(v) \Rightarrow \psi(w)$, and $w$ is not covered
  \end{itemize}
\end{defn}

Notice that, if a vertex is covered, all its descendants are also covered. Moreover, we do not allow a covered vertex to cover another vertex.

\begin{thm}
  If there exists a safe, complete, well-labeled unwinding of program $\mathcal{A}$, then $\mathcal{A}$ is safe.
\end{thm}

This is Theorem 1 from \cite{mcmillan06}.

% describe the standard Impact algorithm
\section{Impact Algorithm}
\label{sec:impact-algorithm}
%

This section describes a semi-algorithm from \cite{mcmillan06}, for building a complete, safe, well-labeled unwinding of a program. The algorithm terminates if the program is unsafe, but may not terminate if it is safe (which is expected, since program safety is undecidable). A non-deterministic procedure with three basic steps is outlined here. The three steps are

\begin{itemize}
  \item \expand, which generates the successors of a leaf vertex (\autoref{alg:expand})
  \item \refine, which refines the labels along a path, labeling an error vertex $\false$ (\autoref{alg:refine})
  \item \cover, which expands the covering relation (\autoref{alg:cover})
\end{itemize}

Each of the three steps preserves well-labeledness of the unwinding. When none of the three steps can produce any change, the unwinding is both safe and complete, so we know that the original program is safe.

\subsection{Algorithm Description}
We now briefly explain how the three main procedures \expand, \refine, and \cover work.

\expand, formally outlined in \autoref{alg:expand}, explores new actions for an uncovered leaf. An uncovered leaf (described in \autoref{defn:labeled-prog-unwinding}) is intuitively one for which all possible successor states have not been explored yet. If the vertex $v \in V$ supplied to \expand is an uncovered leaf, then for each possible action $T$ at the program location $M_v(v)$, a new vertex $w$ is added to the unwinding tree (\autoref{line:expand-add-new-vertex}). This is normally done when a leaf is such that it cannot be covered by any other vertex in the tree, and thus needs further state space exploration.

\begin{algorithm}[ht]
  % Declare functions
  \SetKwFunction{procexpand}{EXPAND}

  % Declare sub-program markers.
  \SetKwProg{myproc}{Procedure}{}{}

  % expand
  \myproc{\procexpand{$v \in V$}:}{
    %
    \If{$v$ is an uncovered leaf}{
        \ForEach{action $(M_v(v),T,m) \in \Delta$}{
        add a new vertex $w$ to $V$ and a new edge $(v,w)$ to $E$; \label{line:expand-add-new-vertex} \\
        set $M_v(w) \leftarrow m$ and $\psi(w) \leftarrow \true$; \\
        set $M_e(v,w) \leftarrow T$;
      }
    }
  }
  \caption{$\expand$: takes as input a vertex $v \in V$ and expands the control flow graph based on all actions available at that vertex.}
  \label{alg:expand}
\end{algorithm}

\refine is the procedure that generates the interpolant, and can potentially mark an error vertex unreachable. Firstly, we note that if \refine succeeds, then $\phi(v)$ must be $\false$ (since $\hat{A}_n$ is always $\false$). Thus, to make the unwinding safe, we have to only apply \refine to every error vertex (\autoref{line:refine-only-error-vertex}) that hasn't yet been marked $\false$. For such an error vertex $v$, \refine queries a decision procedure \cite{mcmillan05} to find an interpolant for the path from root to error vertex (\autoref{line:refine-query-decision-procedure}). If an interpolant is not found, it means that the error state is reachable, and the program is marked unsafe (\autoref{line:refine-program-unsafe}). If an interpolant is found, then it is used to strengthen the predicate labelings at the vertices along the path $\pi$. In particular, if the interpolant formula at a vertex $v_i$ doesn't subsume the current predicate label at $v_i$ (\autoref{line:refine-not-subsumed}), then the current label can be strengthened. Note that if this happens, then the stronger label at $v_i$ may possibly lead it to stop covering some of the vertices it previously covered. This is taken into account by removing all pairs $(\cdot, v_i)$ in the covering relation $\rhd$ (\autoref{line:refine-remove-covering-relations}).

\begin{algorithm}[ht]
  % Declare functions
  \SetKwFunction{procrefine}{REFINE}

  % Declare sub-program markers.
  \SetKwProg{myproc}{Procedure}{}{}

  % expand
  \myproc{\procrefine{$v \in V$}:}{
    %
    \If{$M_v(v) = l_f$ and $\psi(v) \not\equiv \false$}{
      \label{line:refine-only-error-vertex}
      let $\pi = (v_0, T_0, v_1) \cdots (v_{n-1}, T_{n-1}, v_n)$ be the unique path from $\epsilon$ to $v$ \\
      \eIf{$\mathcal{U}(\pi)$ has an interpolant $\hat{A_0},\cdots,\hat{A_n}$}{
          \label{line:refine-query-decision-procedure}
          \For{$i = 0 \cdots n$}{
          let $\phi = \hat{A}_i^{\langle -i \rangle}$ \\
          \If{$\psi(v_i) \nvDash \phi$}{
            \label{line:refine-not-subsumed}
            remove all pairs $(\cdot, v_i)$ from $\rhd$; \label{line:refine-remove-covering-relations} \\
            set $\psi(v_i) \leftarrow \psi(v_i) \land \phi$;
          }
        }
      }
      {
        abort (program is unsafe) \label{line:refine-program-unsafe}
      }
    }
  }
  \caption{$\refine$: takes as input a vertex $v \in V$ at an error location and tags the path from root to $v$ with invariants.}
  \label{alg:refine}
\end{algorithm}

\cover is a simple procedure that takes two vertices $v, w \in V$, and attempts to cover $v$ with $w$, provided $v$ is an uncovered vertex that is not an ancestor of $w$, and they are associated with the same program location (\autoref{line:cover-pre-check}). If the predicate label at $v$ is subsumed by the predicate label at $w$, then it means that $v$ need not be explored further, and can be covered with $w$, by adding the pair $(v,w)$ to $\rhd$. Note that if a vertex is covered, then all its descendants are also automatically covered. Since we don't allow a covered vertex to cover other vertices, each vertex $y$ which is a descendant of $v$, and covers other vertices, must stop covering those vertices (\autoref{line:cover-stop-covering}).

\begin{algorithm}[ht]
  % Declare functions
  \SetKwFunction{proccover}{COVER}

  % Declare sub-program markers.
  \SetKwProg{myproc}{Procedure}{}{}

  % expand
  \myproc{\proccover{$v, w \in V$}:}{
    %
    \If{$v$ is uncovered and $M_v(v) = M_v(w)$ and $v \nvDash w$}{
      \label{line:cover-pre-check}
      \If{$\psi(v) \vDash \psi(w)$}{
        add $(v,w)$ to $\rhd$; \\
        delete all $(x,y) \in \rhd$, s.t. $v \sqsubseteq y$; \label{line:cover-stop-covering}
      }
    }
  }
  \caption{$\cover$: takes as input vertices $v, w \in V$ and attempts to cover $v$ with $w$.}
  \label{alg:cover}
\end{algorithm}

\subsection{Termination}
The \impact algorithm works by repeated applications of \expand, \cover, and \refine. When any of the three procedures don't cause any change in the unwinding tree, the algorithm terminates. To build a well-labeled unwinding, a strategy is required for applying the three procedures above. The most difficult question is when to apply \cover. Covering one vertex can result in uncovering others. Thus, applying \cover non-deterministically may not terminate.

To avoid the possibility of of non-termination, a total order $\preceq$ can be defined on the vertices. This order must respect the ancestor relation. That is, if $v \sqsubset w$, then $v \prec w$. For example, we could define $\preceq$ by a pre-order traversal of the tree, or numbering the vertices in order of creation. After that, \cover is restricted to only those pairs $(v,w)$ for which $w \prec v$. If adding $(v,w)$ to $\rhd$ leads to the removal of $(x,y)$ where $v \sqsubseteq y$, then by transitivity, we must have $v \prec x$. Thus, covering a vertex $v$ can only result in uncovering vertices greater than $v$, implying that we cannot apply \cover indefinitely.

We call a vertex $v$ \textit{closed} if either it is covered, or no arc $(v,w)$ can be added to $\rhd$ while maintaining well-labeledness. It is possible to guarantee that when a vertex is expanded, all of its ancestors are closed, thus making it so that a vertex that could be covered isn't expanded unnecessarily. The \unwind algorithm in \cite{mcmillan06} describes a systematic way of doing this. Forced covering (calling \cover on all nodes of a path after a \refine step) also serves as an optimization. All of these ideas apply directly to the case of heap-manipulating programs.

\section*{Summary}
In this chapter, we presented a framework for modeling programs as state graphs, and an interpolant-based approach to verification that uses program unwindings to explore the state space. In the next chapter, we describe the formalism that allows us to represent and reason about heap memory. This will be useful for defining the verification algorithm for heap-manipulating programs.
% describe the modification of the Impact algorithm for heaps

% describe the standard Impact algorithm
% \section{Heap Impact Algorithm}
\label{ch:heap-impact-algorithm}
%

Building on top of the framework defined in \autoref{sec:impact-algorithm} and \autoref{ch:heap-patterns}, we can define \verifier, by modifying the \impact algorithm to work for heap-manipulating programs. In this section, we first define the three steps of \impact, that is \expand, \cover, and \refine. Then we describe an invariant-learning procedure that retrieves patterns from an Oracle, thereby completing the description of the algorithm.

\begin{algorithm}[ht]
  % Declare functions
  \SetKwFunction{procexpand}{EXPAND}

  % Declare sub-program markers.
  \SetKwProg{myproc}{Procedure}{}{}

  % expand
  \myproc{\procexpand{$v \in V$}:}{
    %
    \If{$v$ is an uncovered leaf}{
        \ForEach{action $(M_v(v),T,m) \in \Delta$}{
        add a new vertex $w$ to $V$ and a new edge $(v,w)$ to $E$; \\
        set $M_v(w) \leftarrow m$ and $\psi(w) \leftarrow 1_D$; \\
        set $M_e(v,w) \leftarrow T$;
      }
    }
  }
  \caption{$\expand$: takes as input a vertex $v \in V$ and expands the control flow graph based on all actions available at that vertex.}
  \label{alg:heap-expand}
\end{algorithm}

\begin{algorithm}[ht]
  % Declare functions
  \SetKwFunction{procrefine}{REFINE}

  % Declare sub-program markers.
  \SetKwProg{myproc}{Procedure}{}{}

  % expand
  \myproc{\procrefine{$v \in V$}:}{
    %
    \If{$M_v(v) = l_f$ and $\psi(v) \not\equiv (\false, 0_D)$}{
      let $\pi = (v_0, T_0, v_1) \cdots (v_{n-1}, T_{n-1}, v_n)$ be the unique path from $\epsilon$ to $v$ \\
      let $\hat{A_0},\cdots,\hat{A_n}$ = \seplearner($\mathcal{U}(\pi)$) \\
      \eIf{$\hat{A_0},\cdots,\hat{A_n}$ is a valid interpolant}{
          \For{$i = 0 \cdots n$}{
          let $\phi = \hat{A}_i^{\langle -i \rangle}$ \\
          \If{$\psi(v_i) \nvDash \phi$}{
            remove all pairs $(\cdot, v_i)$ from $\rhd$; \\
            set $\psi(v_i) \leftarrow \psi(v_i) \land \phi$;
          }
        }
      }
      {
        abort (program is unsafe)
      }
    }
  }
  \caption{$\refine$: takes as input a vertex $v \in V$ at an error location and tags the path from root to $v$ with invariants.}
  \label{alg:heap-refine}
\end{algorithm}

\begin{algorithm}[ht]
  % Declare functions
  \SetKwFunction{proccover}{COVER}

  % Declare sub-program markers.
  \SetKwProg{myproc}{Procedure}{}{}

  % expand
  \myproc{\proccover{$v, w \in V$}:}{
    %
    \If{$v$ is uncovered and $M_v(v) = M_v(w)$ and $v \nvDash w$}{
      \If{$\psi(v) \vDash \psi(w)$}{
        add $(v,w)$ to $\rhd$; \\
        delete all $(x,y) \in \rhd$, s.t. $v \sqsubseteq y$;
      }
    }
  }
  \caption{$\cover$: takes as input vertices $v, w \in V$ and attempts to cover $v$ with $w$.}
  \label{alg:heap-cover}
\end{algorithm}

\section{Postcondition Transforms for Heap Operations}
We first define the operator $\post$, which will be used for generating examples to interact with the Oracle.

% define labeled program unwinding
\begin{defn}
  \label{defn:heap-post-transforms}
  The operator $\post$, which computes the strongest postcondition for a given heap pattern, and action. That is $\post : \heappats \times \mathcal{T} \to \heappats$, where $\mathcal{T}$ is the set of all actions.

  The $\post{*}$ operator can be defined as a repeated application of $\post$ along a given path. More formally, it is $\post{*} : \heappats \times \mathcal{P} \to \heappats$, where $\mathcal{P}$ is a path in the unwinding.
\end{defn}

We also note that the $\post$ operator can be overloaded to work with individual heaps instead of patterns, since single heaps can also be represented using a pattern.

The formal rules for computing $\post$ for each individual action are presented here. We define them for major heap operations that are part of \lang. The formal requirement is that for a heap $h$, pattern $P$, and transition $T$, such that $h \matchedby P$, we must have $\post(h, T) \matchedby \post(P, T)$. Assume that the original pattern is represented by $P = (\nodesnm, \varlblnm, \predlblnm, \edgesnm, \sigma)$, and the pattern after transformation by $\post$ is represented by $P' = (\nodesnm', \varlblnm', \predlblnm', \edgesnm', \sigma')$. We define $P'$ using the definition of $P$, for each possible value of $T$ below. New variables are presented as updates to the values of old variables.

\begin{itemize}
  \item \textbf{ALLOC} ($v := alloc()$): \\
    - $\nodesnm' = \nodesnm \cup \{n\}$ where $n \not \in \nodesnm$ \\
    - $\varlblnm'$ updates $\varlblnm$ such that $\varlblnm'(n, v) = \true$ and $\forall m \neq n, \varlblnm'(m, v) = \false$ \\
    - $\predlblnm'$ updates $\predlblnm$ such that $\forall p, \predlblnm'(n, p) = \maybe$ \\
    - $\edgesnm$ updates $\edgesnm'$ such that $\edgesnm'(n,f,m) = \false$, and $\edgesnm'(m,f,n) = \false$ for all fields $f$ and nodes $m \neq n$ \\
    - $\sigma'$ updates $\sigma$ such that $\sigma'(n) = \true$
  \item \textbf{COPY} ($v1 := v2$): \\
    - $\nodesnm' = \nodesnm$ \\
    - $\varlblnm'$ updates $\varlblnm$ such that $\forall n \in \nodesnm, \varlblnm'(v1, n) = \varlblnm(v2, n)$ \\
    - $\predlblnm' = \predlblnm$ \\
    - $\edgesnm = \edgesnm'$ \\
    - $\sigma' = \sigma$
  \item \textbf{LOAD} ($v1 := v2 \select f$): \\
    - $\nodesnm' = \nodesnm$ \\
    - $\varlblnm'$ updates $\varlblnm$ as follows: \\
      \hspace*{1em} Let $S = \{n \in \nodesnm : \varlblnm(n, v2) = \true \vee \varlblnm(n, v2) = \maybe\}$ \\
      \hspace*{1em} Let $T = \{n \in \nodesnm : \exists s \in S \cdot \edgesnm(s, f, n) = \true \vee \edgesnm(s, f, n) = \maybe\}$ \\
      \hspace*{1em} if $T = \{t\}$ (singleton), then $\varlblnm'(t, v1) = \true$, otherwise $\forall t \in T, \varlblnm'(t, v1) = \maybe$ \\
    - $\predlblnm' = \predlblnm$ \\
    - $\edgesnm = \edgesnm'$ \\
    - $\sigma' = \sigma$
  \item \textbf{STORE} ($v1 \select f := v2$): \\
    - $\nodesnm' = \nodesnm$ \\
    - $\varlblnm' = \varlblnm$ \\
    - $\predlblnm' = \predlblnm$ \\
    - $\edgesnm$ updates $\edgesnm'$ as follows: \\
      \hspace*{1em} Let $S = \{n \in \nodesnm : \varlblnm(n, v1) = \true \vee \varlblnm(n, v1) = \maybe\}$ \\
      \hspace*{1em} Let $T = \{n \in \nodesnm : \varlblnm(n, v2) = \true \vee \varlblnm(n, v2) = \maybe\}$ \\
      \hspace*{1em} $\forall s \in S, t \in T, \edgesnm'(s,f,t) = \maybe$ (and $\true$ if both $S$ and $T$ are singletons) \\
    - $\sigma' = \sigma$
  \item \textbf{PREDICATE} ($\predinstr$): \\
    - $\nodesnm' = \nodesnm$ \\
    - $\varlblnm' = \varlblnm$ \\
    - $\forall n \in \nodesnm, p \in \predvars, \predlblnm'(n,p) = post(p, \predlblnm(n,p), \predinstr)$, where $post$ is a postcondition operator for three-valued predicates \\
    - $\edgesnm = \edgesnm'$ \\
    - $\sigma' = \sigma$
\end{itemize}

\section{Learning Invariants from Positive and Negative Examples}

We note in \autoref{alg:heap-refine} the procedure \seplearner is used, which is the core component of making \impact work for heap-manipulating programs. In this section, we describe this algorithm.

We defined the notion of interpolants in \autoref{sec:interpolants-from-proofs}, and the same applies to \seplearner, which is described in \autoref{alg:invlearner}.

\begin{algorithm}[ht]
  % Declare functions
  \SetKwFunction{procinvlearner}{INVLEARNER}

  % Declare sub-program markers.
  \SetKwProg{myproc}{Procedure}{}{}

  % expand
  \myproc{\procinvlearner{$\mathcal{U}(\pi)$}:}{
    %
    Let $\pi = (l_0, T_0, l_1)(l_1, T_1, l_2) \cdots (l_{n-1}, T_{n-1}, l_n)$ \\
    Set $\hat{A_i} = 1_D, 0 \leq i < n, \hat{A_n} = 0_D$ \\
    Set $H_i^{+} = \{\}, 0 \leq i \leq n$ \\
    Set $H_i^{-} = \{\}, 0 \leq i \leq n$ \\

    \While{$\neg$\isinterpolant($\hat{A},\mathcal{U}(\pi)$)}{
      pick $i \in \{1,2,\cdots, n-1\}$
      $\hat{A_i} = \newcandidate(l_i, \hat{A}, H^{+}, H^{-}, \psi, \mathcal{U}(\pi))$
    }
    \Return $\hat{A_0}, \hat{A_1}, \cdots, \hat{A_n}$
  }
  \caption{$\seplearner$: takes as input an unfolding $\mathcal{U}(\pi)$ of path $\pi$ and attempts to find an invariant for it.}
  \label{alg:invlearner}
\end{algorithm}

\begin{algorithm}[ht]
  % Declare functions
  \SetKwFunction{procisinterpolant}{ISINTERPOLANT}

  % Declare sub-program markers.
  \SetKwProg{myproc}{Procedure}{}{}

  % expand
  \myproc{\procisinterpolant{$\hat{A}, \mathcal{U}(\pi)$}:}{
    %
    \If{$\hat{A_0}, \hat{A_1}, \cdots, \hat{A_n}$ is an interpolant for $\mathcal{U}(\pi)$}{
      \Return $\true$
    }
    \Return $\false$
  }
  \caption{$\isinterpolant$: takes as input candidates $\hat{A}$ and unfolding $\mathcal{U}(\pi)$ of path $\pi$, and checks if $\hat{A}$ represents an interpolant for the unfolding.}
  \label{alg:isinterpolant}
\end{algorithm}

\begin{algorithm}[ht]
  % Declare functions
  \SetKwFunction{procnewcandidate}{NEWCANDIDATE}

  % Declare sub-program markers.
  \SetKwProg{myproc}{Procedure}{}{}

  % expand
  \myproc{\procnewcandidate{$l_i, \hat{A}, H^{+}, H^{-}, \psi, \mathcal{U}(\pi)$}:}{
    %
    Let $\pi = (l_0, T_0, l_1)(l_1, T_1, l_2) \cdots (l_{n-1}, T_{n-1}, l_n)$ \\
    Set $S = \post^{*}(1_D, \pi_{0,i})$ \\
    Set $C = 1_D$ \\
    \While{$\true$}{
      $C = \mathcal{O}(H_i^{+}, H_i^{-})$ \\
      \If{$S \not \entails C$}{
        $H_i^{+} = \{h\} \cup H_i^{+}$ where $h \in S, h \not \in C$ \\
        continue
      }
      \If{$\post(C, T_i) \not \entails \hat{A}_{i+1}$}{
        $\exists h \cdot h \matchedby C \wedge \post(h, T_i) \not \matchedby \hat{A}_{i+1}$ \\
        $H_i^{-} = \{h\} \cup H_i^{-}$ \\
        continue
      }
      break
    }
    \Return $C$
  }
  \caption{$\newcandidate$: takes as input a program location $l_i$, current set of candidates $\hat{A}$, sets of positive and negative examples for each location ($H^{+}, H^{-}$ respectively), map $\psi$, and unfolding $\mathcal{U}(\pi)$ of path $\pi$, and interacts with the Oracle $\mathcal{O}$ to find a new candidate for $l_i$.}
  \label{alg:newcandidate}
\end{algorithm}

While \seplearner is the higher level procedure to find an interpolant, it uses a sub-procedure called \newcandidate as a feedback loop with the Oracle, to accept candidate heap patterns that can be used to construct an interpolant.

\section*{Summary}
This chapter presented \verifier, our algorithm for verifying heap-manipulating programs. \verifier uses an Oracle to perform the interpolation step, which helps in learning heap patterns from positive and negative concrete examples. The next chapter describes the implementation of an interface for one such Oracle - a human user.
% describe the modification of the Impact algorithm for heaps

% describe the standard Impact algorithm
\section{Heap Impact Algorithm}
\label{sec:heap-impact-algorithm}
%

Building on top of the framework defined in \autoref{sec:background}, \autoref{sec:impact-algorithm}, and \autoref{sec:heap-patterns}, we can now modify the \impact algorithm to work for heap-manipulating programs. In this section, we first define the three steps of \impact, that is \expand, \cover, and \refine. Then we describe an invariant-learning procedure that retrieves patterns from an Oracle, thereby completing the description of the algorithm.

\begin{algorithm}[ht]
  % Declare functions
  \SetKwFunction{procexpand}{EXPAND}

  % Declare sub-program markers.
  \SetKwProg{myproc}{Procedure}{}{}

  % expand
  \myproc{\procexpand{$v \in V$}:}{
    %
    \If{$v$ is an uncovered leaf}{
        \ForEach{action $(M_v(v),T,m) \in \Delta$}{
        add a new vertex $w$ to $V$ and a new edge $(v,w)$ to $E$; \\
        set $M_v(w) \leftarrow m$ and $\psi(w) \leftarrow 1_D$; \\
        set $M_e(v,w) \leftarrow T$;
      }
    }
  }
  \caption{$\expand$: takes as input a vertex $v \in V$ and expands the control flow graph based on all actions available at that vertex.}
  \label{alg:heap-expand}
\end{algorithm}

\begin{algorithm}[ht]
  % Declare functions
  \SetKwFunction{procrefine}{REFINE}

  % Declare sub-program markers.
  \SetKwProg{myproc}{Procedure}{}{}

  % expand
  \myproc{\procrefine{$v \in V$}:}{
    %
    \If{$M_v(v) = l_f$ and $\psi(v) \not\equiv (\false, 0_D)$}{
      let $\pi = (v_0, T_0, v_1) \cdots (v_{n-1}, T_{n-1}, v_n)$ be the unique path from $\epsilon$ to $v$ \\
      let $\hat{A_0},\cdots,\hat{A_n}$ = \seplearner($\mathcal{U}(\pi)$) \\
      \eIf{$\hat{A_0},\cdots,\hat{A_n}$ is a valid interpolant}{
          \For{$i = 0 \cdots n$}{
          let $\phi = \hat{A}_i^{\langle -i \rangle}$ \\
          \If{$\psi(v_i) \nvDash \phi$}{
            remove all pairs $(\cdot, v_i)$ from $\rhd$; \\
            set $\psi(v_i) \leftarrow \psi(v_i) \land \phi$;
          }
        }
      }
      {
        abort (program is unsafe)
      }
    }
  }
  \caption{$\refine$: takes as input a vertex $v \in V$ at an error location and tags the path from root to $v$ with invariants.}
  \label{alg:heap-refine}
\end{algorithm}

\begin{algorithm}[ht]
  % Declare functions
  \SetKwFunction{proccover}{COVER}

  % Declare sub-program markers.
  \SetKwProg{myproc}{Procedure}{}{}

  % expand
  \myproc{\proccover{$v, w \in V$}:}{
    %
    \If{$v$ is uncovered and $M_v(v) = M_v(w)$ and $v \nvDash w$}{
      \If{$\psi(v) \vDash \psi(w)$}{
        add $(v,w)$ to $\rhd$; \\
        delete all $(x,y) \in \rhd$, s.t. $v \sqsubseteq y$;
      }
    }
  }
  \caption{$\cover$: takes as input vertices $v, w \in V$ and attempts to cover $v$ with $w$.}
  \label{alg:heap-cover}
\end{algorithm}

\subsection{Postcondition Transforms for Heap Operations}
TODO: This section needs to be completed.

% define labeled program unwinding
\begin{defn}
  \label{defn:heap-post-transforms}
  We define the operator $\post$, which computes the strongest postcondition for a given heap pattern, and action. That is $\post : \heappats \times \mathcal{T} \to \heappats$, where $\mathcal{T}$ is the set of all actions.
\end{defn}

The formal rules for computing $\post$ for each individual action are presented in \autoref{fig:post-transforms}. (TODO)

\subsection{Learning Invariants from Positive and Negative Examples}

We note in \autoref{alg:heap-refine} the procedure \seplearner is used, which is the core component of making \impact work for heap-manipulating programs. In this section, we describe this algorithm.

We defined the notion of interpolants in \autoref{sec:interpolants-from-proofs}, and the same applies to \seplearner, which is described in \autoref{alg:invlearner}.

\begin{algorithm}[ht]
  % Declare functions
  \SetKwFunction{procinvlearner}{INVLEARNER}

  % Declare sub-program markers.
  \SetKwProg{myproc}{Procedure}{}{}

  % expand
  \myproc{\procinvlearner{$\mathcal{U}(\pi)$}:}{
    %
    \If{$v$ is uncovered and $M_v(v) = M_v(w)$ and $v \nvDash w$}{
      \If{$\psi(v) \vDash \psi(w)$}{
        add $(v,w)$ to $\rhd$; \\
        delete all $(x,y) \in \rhd$, s.t. $v \sqsubseteq y$;
      }
    }
  }
  \caption{$\seplearner$: takes as input an unfolding $\mathcal{U}(\pi)$ of path $\pi$ and attempts to find an invariant for it.}
  \label{alg:invlearner}
\end{algorithm}
% Describe heap patterns

% \section{Heap Patterns}
\label{ch:heap-patterns}
%

To extend the \impact algorithm to heaps, we first define a framework for representing and reasoning about heaps.


In this section, we first define a standard language that performs low-level
memory operations to update linked data structures
(\autoref{sec:lang}).
%
We then review definitions of three-valued structures introduced in
previous work \cite{sagiv02}, which we use to formulate patterns over
program heaps (\autoref{sec:patterns}).

% Define a toy imperative language.
\section{Language Definition}
\label{sec:lang}
%

In this section, we define the syntax (\autoref{sec:syntax}) and
semantics (\autoref{sec:semantics}) of our subject language \lang.

% Define the language syntax.
\subsection{Syntax}
\label{sec:syntax}
% Define the syntax of a program.
\begin{figure}
  \centering
  \begin{align}
    % A program is a sequence of statements.
    \lang \assign & (\locs: \instrs)^{*} \label{line:prog} \\
    % A statement is a predicate instruction or a heap instruction.
    \instrs \assign & \predinstr\ |\ \heapinstr \label{line:instrs} \\
    % A predicate instruction:
    \predinstr \assign &
    % updates predicates according to an operation.
    \predvars \assign \predvars\ \predops\ \predvars
    %
    \label{line:bool-op} \\
    % or tests equalities of heap variables,
    |\ & \predvars \assign (\heapvars = \heapvars)
    %
    \label{line:test-eq} \\
    % or branches.
    |\ & \branch\ \predvars, \locs, \locs \label{line:branch} \\
    % A heap instruction: allocates a new cell:
    \heapinstr \assign & \heapvars \assign \alloc()
    %
    \label{line:alloc} \\
    % or a copy,
    |\ & \heapvars \assign \heapvars \label{line:copy} \\
    % or a load,
    |\ & \heapvars \assign \heapvars \select \fields
    %
    \label{line:load} \\
    % or a store.
    |\ & \heapvars \select \fields \assign \heapvars
    %
    \label{line:store}
  \end{align}
  \caption{Syntax of heap-updating programs, $\lang$.
    %
    The spaces of control locations, predicate variables, heap
    variables, and fields are denoted $\locs$, $\predvars$,
    $\heapvars$, and $\fields$, respectively.
  }
  \label{fig:syntax}
\end{figure}

% Walk through the full syntax of programs.
A \lang program is a sequence of instructions that operate on a fixed
set of predicate variables and pointers to heap objects.
%
The syntax of \lang is given in \autoref{fig:syntax} for fixed finite
sets of control locations $\locs$, predicate variables $\predvars$,
heap variables $\heapvars$, and heap fields $\fields$.
%
A program is a sequence of instruction, each labeled with a control
location (\autoref{line:prog}).
%
An instruction either updates the program's predicate variables or
heap variables (\autoref{line:instrs}).
%
An instruction that updates predicate variable either stores in a
predicate variable the result of a Boolean operation
(\autoref{line:bool-op}),
%
an equality test on heap cells (\autoref{line:test-eq}),
%
or branches control based on the value in a predicate variable
(\autoref{line:branch}).
%
An instruction that updates heap variables either allocates a new heap
cell (\autoref{line:alloc}),
%
copies the heap cell from one pointer variable to another
(\autoref{line:copy}),
%
loads a heap cell into a pointer variable (\autoref{line:load}),
%
or stores a heap cell as a child of a cell in a pointer variable
(\autoref{line:store}).

% Define the semantics.
\subsection{Semantics}
\label{sec:semantics}
%
A \lang program updates a state, which consists of a control location,
evaluation of predicate variables, and a heap, which is a graph with
labeled edges.
%

Also refer to \autoref{fig:semantics}.

% Transfer functions for each instruction.
\begin{figure*}
  \centering
  \begin{gather*}
    % Allocation:
    \inference[ALLOC]
    %
    { n \notin N & N' = \add{n}{N} \\
      %
      V' = \upd{V}{\cc{h}}{n}
      & F' = F[ \{ (n, f) \mapsto \nil \}_{f \in \fields} ]
    }
    %
    { \langle (C, V, F), \code{h \assign \alloc()} \rangle \step
      %
      (N', V', F')
    } \rulebuffer
    % Copy:
    \inference[COPY]
    %
    {  V' = \upd{V}{\cc{g}}{V(\cc{h})}
    }
    %
    { \langle (C, V, F), \code{g \assign h} \rangle \step
      %
      (C, V', F)
    } \\[\rowspace]
    % Load:
    \inference[LOAD]
    { V' = \upd{V}{\code{p}}{F(V(\code{q}), \code{f})}
    }
    %
    { \langle (C, V, F), \code{p \assign q \select f} \rangle \step
      %
      (C, V', F)
    } \rulebuffer
    % Store:
    \inference[STORE]
    { F' = \upd{F}{(\cc{p}, \cc{f})}{V(\cc{q})}
    }
    { \langle (C, V, F), \code{p \select f \assign q } \rangle
      \step
      %
      (C, V, F')
    }
  \end{gather*}
  \caption{Inference rules that define $\heapstep$, the transition
    relation over heaps and heap updates.}
  \label{fig:semantics}
\end{figure*}

% Define the language of patterns.
\section{Heap pattern Language}
\label{sec:patterns}

\begin{defn}
  \label{defn:states}
  % Define space of heaps.
  A \lang \emph{heap} is a triple $(\cellsnm, \heapvarnm,
  \fieldmapnm)$, where
  %
  \begin{enumerate}
  \item
    % Space of cells.
    For the countable universe of heap cells, denoted $\celluni$,
    $\cellsnm \subseteq \celluni$ is a finite set of cells that contains
    a distinguished cell $\nil \in \celluni$.
  \item
    % Heap-variable labeling:
    $\heapvarnm: \heapvars \to \cellsnm$ maps each heap variable to
    the cell that it stores.
    %
    We denote the space of evaluations of heap variables as
    $\heapevals = \heapvars \to \celluni$.
  \item
    % Field mapping:
    $\fieldmapnm: \cellsnm \times \fields \to \cellsnm$ maps each cell
    $c \in \cellsnm$ and field $\code{f} \in \fields$ to the child of
    $c$ at $\code{f}$.
    %
    We denote the space of field maps as $\fieldmaps = \celluni \times
    \fields \to \celluni$.

  \item
    % Predicate labeling:
    $\predlabel: \cellsnm \to 2^{\predvars}$ maps each cell
    $c \in \cellsnm$ to an assignment of Boolean values to $\predvars$.
    %
    We denote the space of predicate labeling functions as
    $\predlabelfns = \celluni \to 2^{|\predvars|}$.

  \end{enumerate}
  %
  We denote the space of heaps as $\heaps = \pset(\celluni) \times
  \heapevals \times \fieldmaps \times \predlabelfns$, and the space of
  characteristic functions of languages of heaps as $\chars = \heaps
  \to \bools$.
\end{defn}

% Define patterns, matching.
A heap pattern is a labeled graph whose nodes and edges model the
cells and fields of potentially-many heaps.
%
The nodes and edges of a pattern are annotated with three-valued truth
values that represent if all cells modeled by a node either definitely
are, definitely are not, or may be (1) stored in a given heap variable
or (2) connected by a heap field.
%
The heap patterns are based on three-valued structures introduced in
previous work on shape analysis~\cite{sagiv02}.
%
\begin{defn}
  \label{defn:pattern}
  %
  Let the domain of \emph{three-valued truth values} be $\threevals =
  \{ \true, \false, \maybe \}$.
  % Define structure of a heap pattern.
  A \emph{heap pattern} is a labeled graph $(\nodesnm, \varlblnm,
  \predlblnm, \edgesnm, \sigma)$, where:
  %
  \begin{enumerate}
  \item
    % Nodes:
    For the countable universe of nodes $\nodeuni$, $\nodesnm
    \subseteq \nodeuni$ is a finite set of nodes.
  \item
    % Heap-variable labeling:
    $\varlblnm: \nodesnm \times \heapvars \to \threevals$ is a
    \emph{heap-variable labeling}.
    %
    We denote the space of heap-variable labelings as $\varlbls =
    \nodeuni \times \heapvars \to \threevals$.
  \item
    % Predicate labeling:
    $\predlblnm: \nodesnm \times \predvars \to \threevals$ is a
    \emph{predicate-variable labeling}.
    %
    We denote the space of predicate-variable labelings as $\predvarlbls =
    \nodeuni \times \predvars \to \threevals$.
  \item
    % Edges:
    $\edgesnm: \nodesnm \times \fields \times \nodesnm \to \threevals$ is a set of
    labeled \emph{edges}.
    %
    We denote the space of labeled edges as $\lbledges = \nodeuni
    \times \fields \times \nodeuni$.
  \item
    % Summary:
    $\sigma: \nodesnm \to \bools$ is a summary-labeling for nodes. A node is a summary node if it can be used to represent more than one concrete node.
  \end{enumerate}
  %
  We denote the class of all heap patterns as $\heappats =
  \pset(\nodeuni) \times \varlbls \times \predvarlbls \times \lbledges$.
\end{defn}

Additionally, we define the universal and empty patterns ($1_D$ and $0_D$ respectively). These are analogous to $\true$ and $\false$.

% Example: the heap patterns for altlist.
Program in a language with low-level memory updates, such as \altlist
(\autoref{sec:ex-program}), can be modeled as \lang programs if a
verifier is provided with a bounded set of ``relevant'' predicates $R$
over heap cells.
%
In such a case, the verifier can simply model the predicates in $R$ as
heap fields.
%
To simplify the presentation of our analysis, we assume that a
separate program analysis has inferred such a set of relevant
predicates, and that such predicates are already modeled as \lang
fields.
%


\begin{ex}
  \label{ex:heap-pats-defn}
  % Reintroduce the pattern.
  Pattern $P$ (\autoref{sec:ex-patterns}, \autoref{fig:alt-pattern})
  is a heap pattern
  % TODO: over the predicates [...]
  that contains two summary nodes $s_0$ and $s_1$, and one non-summary
  node $n$.
  % Walk through the unary predicates for each node.
  For summary node $s_0$, predicate \texttt{data!=d} maps to $\true$
  and all other predicates map to $\false$
  %
  For summary node $s_1$, predicate \texttt{data=d} maps to $\true$,
  predicate \lsnm maps to $\maybe$, and all other predicates map to
  $\false$.
  %
  For concrete node $n$, predicate \texttt{EqNil} maps to $\true$ and
  all other predicates map to $\false$.

  % Walk through the binary predicates.
  The edges $(s_0, s_1)$, $(s_0, n)$, $(s_1, s_0)$, and $(s_1, n)$ map
  on field \nextnm to $\maybe$.
  %
  All other edges on all other fields map to $\false$.
\end{ex}

% Define the heap pattern.
Each heap $\heapnm$ defines a heap pattern that represents exactly
$\heapnm$.
%
\begin{defn}
  \label{defn:heap-pat}
  For each heap $\heapnm = (\cellsnm, \heapvarnm, \fieldmapnm)$, the
  \emph{concrete pattern} of $\heapnm$ is $\heapgph_{\heapnm} =
  (\nodesnm_{\heapnm}, \varlblnm_{\heapnm}, \edgesnm_{\heapnm})$,
  where:
  \begin{enumerate}
  \item
    % Nodes:
    Each node in $\heapgph_{\heapnm}$ is a cell in $\cellsnm$.
    %
    I.e., $\nodesnm_{\heapnm} = \cellsnm$.
  \item
    % Variable labeling:
    Each heap-variable binding in $\heapvarnm$ defines a node labeling
    in $\varlblnm_{\heapnm}$.
    %
    I.e., for each node $c \in \cellsnm$ and heap variable $\cc{p} \in
    \heapvars$, if $\heapvarnm(\cc{p}) = c$, then
    $\varlblnm_{\heapnm}(c, \cc{p}) = \true$, and otherwise,
    $\varlblnm_{\heapnm}(c, \cc{p}) = \false$.
  \item
    % Edges:
    Each field binding in $\fieldmapnm$ defines an edge in
    $\edgesnm_{\heapnm}$.
    %
    I.e., for all cells $\cellnm, \cellnm' \in \cellsnm$ and each
    field $\fieldnm \in \fields$, if $\cellnm' = \fieldmapnm(\cellnm,
    \fieldnm)$, then $\edgesnm_{\heapnm}(\cellnm, \fieldnm, \cellnm')
    = \true$, and otherwise, $\edgesnm_{\heapnm}(\cellnm, \fieldnm,
    \cellnm') = \false$.
  \end{enumerate}
\end{defn}
% Call back to a pattern for a concrete heap.
\begin{ex}
  \label{ex:concrete-heaps-defn}
  \autoref{sec:ex-patterns}, \autoref{fig:alt-pattern} contains
  concrete patterns for six heaps: three alternating heaps and three
  non-alternating heaps.
\end{ex}

\begin{defn}
  \label{ex:pat-entails-defn}
  Given two heap patterns $P_1$ and $P_2$, we say that $P_1 \entails P_2$ ($P_1$ entails $P_2$) if every heap represented by $P_1$ is also represented by $P_2$.
\end{defn}

The algorithm for checking entailment is described in \autoref{sec:entailment-algorithm}

% Define matching.
The entailment relation over heap patterns formulates under both (1)
under what conditions a heap pattern describes a heap and (2) under
what conditions all of the heaps described by one heap pattern are
described by another pattern.
%
\begin{defn}
  \label{defn:match}
  % Define the precision ordering.
  Let the \emph{information-precision ordering} $\precord \subseteq
  \threevals \times \threevals$ be such that $\true \precord \maybe$
  and $\false \precord \maybe$.

  % Define entailment over heap patterns.
  For all heap patterns $\patnm, \patnm' \in \heappats$ with $\patnm =
  (\nodesnm, \varlblnm, \edgesnm)$ and $\patnm' = (\nodesnm',
  \varlblnm', \edgesnm')$, $\patnm$ \emph{entails} $\patnm'$, denoted
  $\patnm \matchedby \patnm'$, if there is a map $\matchnm: \nodesnm
  \to \nodesnm'$ such that:
  %
  \begin{itemize}
  \item
    % Variable labelings are preserved.
    $\matchnm$ embeds heap-variable assignments.
    %
    I.e., for each pattern node $\nodenm \in \nodesnm$,
    $\varlblnm(\nodenm) \precord \varlblnm'(\matchnm(\nodenm))$.
  \item
    % Fields labelings are preserved.
    $\matchnm$ embeds field labelings.
    %
    I.e., for all pattern nodes $\nodenm_0, \nodenm_1 \in \nodesnm$
    and fields $\fieldnm \in \fields$, $\edgesnm(\nodenm_0, \fieldnm,
    \nodenm_1) \precord \edgesnm'(\matchnm(\nodenm_0), \fieldnm,
    \matchnm(\nodenm_1))$.
  \end{itemize}
\end{defn}
% Define heap modeling in terms of entailment.
For each heap $\heapnm \in \heaps$ and pattern $\patnm \in \heappats$,
if the concrete pattern (\autoref{defn:heap-pat}) of $\heapnm$ entails
$\patnm$, then $\heapnm$ is \emph{modeled} by $\patnm$, which we
alternatively denote as $\heapnm \matchedby \patnm$.
%
For any two heap patterns $\patnm_0, \patnm_1 \in \heappats$ and heap
$\heapnm \in \heaps$, if $\patnm_0 \entails \patnm_1$ and $\heapnm
\matchedby \patnm_0$, then $\heapnm \matchedby \patnm_1$.
%
\begin{defn}
  \label{ex:pat-entails}
  In \autoref{sec:ex-patterns}, \autoref{fig:alt-pattern}, each of the
  patterns $A_0$, $A_1$, and $A_2$ for an alternating heap entails the
  pattern $\patnm$.
  %
  A matching from $A_1$ to $\patnm$ is depicted as dotted arrows from
  the nodes of $A_1$ to the nodes of $\patnm$.
\end{defn}

% Describe the entailment algorithm.
\section{Pattern Entailment Algorithm}
\label{sec:entailment-algorithm}

TODO: The entailment algorithm is based on simulation checks, and is very similar. It is described here.

By definition, every pattern $P \neq 0_D$ satisfies the following properties:
\begin{itemize}
  \item $P \entails 1_D$
  \item $P \not \entails 0_D$
  \item $0_D \entails P$
\end{itemize}

% Describe program modeling for heap programs
\section{Modeling Heap Programs}
\label{sec:modeling-heap-programs}

We can now use the heap patterns described in \autoref{defn:pattern} to model programs for our modified \impact algorithm for heap-manipulating programs. Heap programs are modeled similarly to other programs, as described in \autoref{sec:modeling-programs}. For the purposes of the \impact algorithm, the key difference is in how program unwindings are constructed. Instead of labeling vertices of the unwinding with FOL formulas, we use heap patterns.

% define labeled program unwinding
\begin{defn}
  \label{defn:labeled-heap-prog-unwinding}
  A labeled unwinding of a program $\mathcal{A} = (\Lambda, \Delta, l_i, l_f)$ is a triple $(U, \psi, \rhd)$, where

  \begin{itemize}
    \item $U = (V, E, M_v, M_e)$ is an unwinding of $\mathcal{A}$
    \item $\psi : V \to \heappats$ is called the vertex labeling, and
    \item $\rhd \subseteq V \times V$ is called the covering relation
  \end{itemize}

  Note that the major difference is that now vertices $v \in V$ contain a heap pattern, instead of an FOL formula.
\end{defn}

Each heap pattern contains a special node \nilconst, which is used to represent the location of pointers that aren't allocated to a specific node.
% technical approach: technical details of our approach
\section{Technical Approach}
\label{sec:approach}
%
In this section, we give the design (\autoref{sec:technique}) and key
correctness properties (\autoref{sec:properties}) of a \reachprob
verifier, \verifier.

% Give the design of the verifier.
\subsection{\verifier Design}
\label{sec:technique}
In this section, we give the design of \verifier.
%
For a given program $\prognm$, \verifier attempts to derive inductive
invariants that prove the correctness of $\prognm$ that are
represented as heap shapes, using operations given in
\autoref{sec:shapes-defn}.
%
In particular, \verifier attempts to construct a proof represented as
a partial unrolling of the control paths of $\prognm$, annotated with
heap shapes as invariants (\autoref{sec:unwindings}).
%
To construct an unwinding tree that represents a valid proof of the
safety of $\prognm$ (\autoref{sec:core}), \verifier uses an operation
\refine (\autoref{sec:refine}), which refines candidate invariants of
$\prognm$.

% Define shape domains, give examples.
\subsection{Shape Domains}
%
In this section, we define \emph{shape domains} as an abstract space
equipped with operations needed by our analysis
(\autoref{sec:shapes-defn}).
%
We then describe abstractions for shape analysis presented in previous
work can be viewed as shape domains (\autoref{sec:shape-instances}).
%
\subsubsection{Definition of shape domains}
\label{sec:shapes-defn}
%
A shape domain is a space in which each element describes a set of
program heaps.
% Define shape domains:
\begin{defn}
  \label{defn:shape-domain}
  %
  A shape domain is a tuple $(\shapes, \in, \noheaps, \allheaps,
  \chksubsumes, \abspost, \optimal)$, where:
  %
  \begin{itemize}
  \item
    % Space of shapes.
    $\shapes$ is the space of heap \emph{shapes}.
  \item
    % Membership:
    $\in \subseteq \heaps \times \shapes$ is the \emph{membership}
    relation.
    %
    For each shape $S \in \shapes$, we denote the set of shapes that
    are members of $S$ (i.e., the \emph{language} of $S$) as
    $\languageof(S)$.
  \item
    % No heaps:
    $\noheaps \in \shapes$ is the \emph{empty} shape.
    %
    For each heap $h \in \heaps$, $h \notin \noheaps$.
  \item
    % All heaps:
    $\allheaps \in \shapes$ is the \emph{universal} shape.
    %
    For each heap $h \in \heaps$, $h \in \allheaps$.
  \item
    % Subsumption:
    Let $S_0, S_1 \in \shapes$ be shapes.
    %
    If for each heap $h \in \heaps$ such that $h \in S_0$ it holds
    that $h \in S_1$, then $S_0$ is \emph{subsumed} by $S_1$, denoted
    $S_0 \subsumedby S_1$.
    %
    If $S_0$ subsumes $S_1$ and $S_1$ subsumes $S_0$, then we say that
    $S_0$ is \emph{equivalent} to $S_1$.

    % Define the subsumption check.
    $\chksubsumes: \shapes \times \shapes \partto
    \add{\heaps}{\isvalid}$ is an effective procedure such that for
    all shapes $S_0, S_1 \in \shapes$, if $S_0 \subsumedby S_1$, then
    $\chksubsumes(S_0, S_1) = \isvalid$;
    %
    otherwise, $\chksubsumes(S_0, S_1) = h \in \heaps$ such that $h
    \in S_0$ and $h \notin S_1$.
  \item
    % Transformers:
    $\abspost: \instrs \to \shapes \to \shapes$ maps each instruction
    to an abstract forward transformer over shapes.
    %
    For each instruction $\code{i} \in \instrs$, heap $h \in \heaps$,
    and shape $S \in \shapes$, if $h \in S$, then
    $\post[\code{i}](h) \in \abspost[\code{i}](S)$.
  \item
    % Optimality:
    $\optimal \subseteq \shapes \times \shapes$ is the
    \emph{optimality} partial ordering over shapes.
    %
    For all shapes $S_0, S_1 \in \shapes$, we denote the fact that
    $S_0$ is \emph{more optimal} than $S_1$ as $S_0 \optimal S_1$.
  \end{itemize}
\end{defn}

\subsubsection{Instances}
\label{sec:shape-instances}
%
Previous work on shape analysis has presented several abstract domains
that can be extended to form shape domains.
%
\paragraph{Three-valued structures}
%
Three-valued Logic Analysis (TVLA) represents sets of program states
as three-valued structures~\cite{sagiv02}.
%
TVLA uses a domain of shapes that are three-valued structures,
extended with a bottom element, which serves as an empty element.
%
Both the membership function and subsumption procedures are
defined using an embedding relation from two-valued structures, which
represent memory heaps, and three-valued structures to tree-valued
structures.
%
Abstract transformers are represented as predicate transformers
specified as first-order logical formulas.

% Optimality:
While previous work on TVLA does not explicitly define optimality
orderings over shapes, orderings that naturally describe the
properties of interest for a shape analysis include the subsumption
relation and minimality over equivalent patterns.

\paragraph{Classes of graph automata}
%
Automata for various classes of graphs can represent sets of heaps.
%
In particular, word automata can be naturally extended to represent
heaps that are bounded tuples of lists, and tree automata can be
naturally extended to represent heaps that are bounded tuples of
trees.
%
Membership and subsumption can be implemented for such domains with
standard operations over automata, and abstract transforms can also be
defined.

% Optimality.
One natural optimality ordering over such shapes orders graph automata
$A_0$ as more optimal than $A_1$ if $A_0$ contains fewer states than
$A_1$.

\subsubsection{Automatically-Learnable Shape Domains}
%
A shape domain $D$ is \emph{learnable} if there is an automatic
procedure that for each language $L$ of graphs, can learn a shape in
$D$ whose language is $L$ using access to only the characteristic
function of $L$ and a function that validates elements in $D$ as
having language $L$.
% Define a learnable language.
\begin{defn}
  \label{defn:learnable}
  %
  A \emph{learnable shape domain} is a pair $(D, \shapelearner)$,
  where:
  %
  \begin{itemize}
  \item
    % Shape domain:
    $D$ is a shape domain.
  \item
    %
    Let $\validatornm: \shapes \to \{ \isvalid \} \union \bools \times
    \heaps$ be a \emph{validator} for $L$.
    %
    I.e., for each shape $S \in \shapes$, if $\languageof(S) = L$,
    then $\validatornm(S) = \isvalid$;
    %
    otherwise, if $\languageof(S) \not\subseteq L$, then
    $\validatornm(S) = (\true, h)$, where $h \in \languageof(S)$ and
    $h \notin L$;
    %
    otherwise (if $L \not\subseteq \languageof(S)$), then
    $\validatornm(S) = (\false, h)$, where $h \notin \languageof(S)$
    and $h \in L$.
    %
    We denote the space of all validators for $D$ as $\validators_D =
    \shapes_D \to \{ \isvalid \} \union \bools \times \heaps$.

    % Define the learner.
    The \emph{shape learner} $\shapelearner: \chars \times
    \validators_D \to \shapes_D$ is such that for each language $L
    \subseteq \heaps$, if $\charnm_L \in \chars$ is a characteristic
    function for $L$ and $\validatornm_L \in \validators_D$ is a
    validator for $L$, then $\shapelearner(\charnm_L, \validatornm_L)$
    is an optimal shape in $\shapes_D$ with language $L$.
  \end{itemize}
\end{defn}

% Instances.
\paragraph{Instances}
% We know how to do it for graph automata.
The shape domains of deterministic word and tree automata ordered by
minimality of size can be extended to learnable shape domains, using
algorithms presented for learning regular languages of
words~\cite{angluin87} and trees~\cite{besombes07}.

% But not three-valued structures.
However, there are no known results that provide active learners for
three-valued structures.

% Define oracles for learning under partial information.
\subsection{Learning Oracles}
%
We will present two shape analyses, each of which attempts to prove
the safety of a given program by querying an oracle for a learning
problem over heaps and shapes.
%
To define the queries answered by oracles in both classes, it will be
useful to first define a relevant inference problem over shape
domains.
% Define partial characterization function, valid refinement.
\begin{defn}
  % Define partial characterization functions.
  A \emph{partial characterization} function is a function $\partchar:
  \heaps \to \add{\bools}{\unknown}$.
  %
  We denote the space of all partial characterization functions as
  $\partchars = \heaps \to \add{\bools}{\unknown}$.

  % Define valid refinement.
  For shape domain $D$, a \emph{valid refinement} of $\partchar$ is a
  shape $S \in \shapes_D$ such that for each heap $h \in \heaps$,
  \begin{itemize}
  \item
    % Positive matches:
    If $\partchar(h) = \true$, then $h \in S$.
  \item
    % Negative matches:
    If $\partchar(h) = \false$, then $h \notin S$.
  \end{itemize}

  % Define the inference problem.
  A solution to the \emph{optimal refinement inference problem}
  $\optrefine(\partchar, D)$ is a shape $S \in \shapes_D$ such that:
  %
  \begin{enumerate}
  \item $S$ is a valid refinement of $\partchar$.
  \item $S$ is at least as optimal as all valid refinements of
    $\partchar$ in $D$.
  \end{enumerate}
\end{defn}

% Define synthesizer oracle.
A synthesizing oracle for shape domain $D$ takes a partial
characterization function $\chi$ over heaps and directly solves the
optimal refinement inference problem defined by $\chi$.
%
\begin{defn}
  \label{defn:syn-oracle}
  For each shape domain $D \in \shapedomains$, a
  \emph{shape-synthesizing oracle} is a function $O: \partchars \to
  D$ such that for each partial characterization function $\chi
  \in \partchars$, $O(\chi)$ is a solution to $\optrefine(\chi, D)$.
  %
  We denote the space of all synthesis oracles for $D$ as
  $\synoracles_D = \partchars \to D$.
\end{defn}

% Define characterizing oracle.
For shape domain $D$, a shape characterization oracles takes a partial
characterization function $\chi$ and implements the characteristic
function of a solution to the optimal refinement inference problem
defined by $\chi$.
%
\begin{defn}
  \label{defn:char-oracle}
  %
  For each shape domain $D \in \shapedomains$, a
  \emph{characterization oracle} is a function $O: \partchars \to
  \chars$ such that for each partial characterization function $\chi
  \in \partchars$, $O(\chi)$ is the characteristic function of a
  solution to $\optrefine(\chi, D)$.
  %
  We denote the space of all characterization oracles for $D$ as
  $\charoracles_D = \partchars \to \chars$.
\end{defn}

% Define unwinding trees, give key proof properties.
\subsection{Unwinding Trees}
\label{sec:unwindings}
%
\verifier models a program as a tree of program paths annotated with
heap shapes as invariants of heaps reached by all runs of the path.
\begin{defn}
  \label{defn:shape-tree}
  %
  For a fixed shape domain $D$, a \emph{shape tree} is a four-tuple
  $(\nodesnm, \edgesnm, \shapesnm, \covernm)$, where:
  %
  \begin{enumerate}
  \item
    % Nodes.
    For the countable universe of tree nodes $\nodeuni$, $\nodesnm
    \subseteq \nodeuni$ is the set of \emph{nodes}.
  \item
    % Tree edges.
    $\edgesnm \subseteq \nodesnm \times \nodesnm$ is the set of
    \emph{edges}, where the graph $(\nodesnm, \edgesnm)$ is a tree.
    %
    We denote the ancestor relation over nodes in $\nodesnm$ as $\anc
    \subseteq \nodesnm \times \nodesnm$.
  \item
    % Patterns.
    $\shapesnm: \nodesnm \to \shapes_D$ maps each node to an invariant
    represented as a heap shape (\autoref{defn:shape-domain}).
  \item
    % Define the cover relation.
    $\covernm \subseteq \nodesnm \times \nodesnm$ is the \emph{cover}
    relation.
    %
    For all nodes $m, n \in \nodesnm$, if $(m, n) \in \covernm$, then
    $m$ is a leaf and $\shapesnm(m) \matchedby \shapesnm(n)$.
    %
    In such a case, we say that $m$ is \emph{covered}.
    %
    A tree is \emph{complete} if all of its leaves are covered.
  \end{enumerate}
  % Define space of shape trees.
  We denote the space of shape trees as $\shapetrees = \pset(\nodeuni)
  \times (\nodeuni \times \nodeuni) \times (\nodesnm \to \shapes_D)
  \times (\nodeuni \times \nodeuni)$.
\end{defn}
% Notation for projecting components of an unwinding tree.
For unwinding tree $\treenm$, we denote the components of $\treenm$ as
$\nodesnm_{\treenm}$, $\edgesnm_{\treenm}$, $\shapesnm_{\treenm}$, and
$\covernm_{\treenm}$.
% Example: call back to the shape tree for altlist.
\begin{ex}
  \label{ex:pat-tree-defn}
  %
  \autoref{sec:ex-tree}, \autoref{fig:alt-unwinding} depicts a shape
  tree $(\nodesnm, \edgesnm, \shapesnm, \covernm)$ that models some
  executions of \altlist (formulated precisely below).
  %
  The nodes $\nodesnm$ are depicted as circles, and the edges
  $\edgesnm$ between nodes are depicted as solid edges.
  %
  For each node $n \in \nodesnm$, the shape $\shapesnm(\nodenm)$ is
  provided as an annotation.
  %
  The only cover edge in $\covernm$ is depicted as a dashed edge from
  node $6$ to node $5$.
\end{ex}

% Define unwindings of a program: unwinding tree with maps to program
% objects.
Unwinding trees model the possible runs of programs.
\begin{defn}
  \label{defn:unwinding-tree}
  For program $\prognm \in \lang$, an \emph{unwinding tree} is a pair
  $(\pattreenm, \locmapnm)$, where:
  %
  \begin{itemize}
  \item
    % Pattern tree.
    $\pattreenm = (\nodesnm, \edgesnm, \shapesnm, \covernm) \in
    \shapetrees$ is a shape tree (\autoref{defn:shape-tree}).
  \item
    % Location map.
    $\locmapnm: \nodesnm \to \locs$ is a map from each node of
    $\pattreenm$ to a control location such that for each uncovered
    node $n \in \nodesnm$ and each control-flow edge $(\locmapnm(n),
    \instrnm, \locnm') \in \edgesnm$, there is some tree edge $(n, n')
    \in \edgesnm$ such that $\post[\instrnm](\shapesnm(n)) \matchedby
    \shapesnm(n')$.
    %
    We denote the space of location maps as $\locmaps = \nodesnm \to
    \locs$.
  \end{itemize}
  %
  We denote the space of unwinding trees as $\unwindtrees =
  \shapetrees \times \locmaps$.
\end{defn}

% Describe the roles of unwinding trees
Unwinding trees constructed by \verifier to attempt to prove that a
given error location is unreachable in a given program.
%
In particular, for each program $\prognm$ with unwinding tree $\treenm
= ((\nodesnm, \edgesnm, \shapesnm, \covernm), \locmapnm)$ and error
control location $\errloc \in \locs$, let $\treenm$ be \emph{safe} if
for each node $\nodenm \in \nodesnm$ such that $\locmapnm(\nodenm) =
\errloc$ (i.e., each node $\nodenm$ that \emph{models} $\errloc$), the
$\patnm(\nodesnm)$ is not matched by any heap.
%
If $\prognm$ has a complete and safe unwinding tree, then $\locnm$ is
not reachable in any run of $\prognm$ (this fact is established and
used in the proof of soundness of \verifier, given in
\autoref{sec:soundness}).

% Example: shape tree from the overview as an unwinding tree of
% altlist.
\begin{ex}
  \label{ex:altlist-unwinding}
  The unwinding tree $\treenm$ depicted in \autoref{sec:ex-tree},
  \autoref{fig:alt-unwinding} is an unwinding tree for $\prognm$, with
  a location map from nodes $0$, $2$, $4$, and $6$ to line 5 of
  \altlist, and from nodes $1$, $3$, and $5$ to line $9$ of \altlist.
  %
  $\treenm$ is not complete, because only the leaf node $6$ is
  covered.
\end{ex}

% Give the core algorithm for the verifier.
\subsection{\verifier Core}
%
\label{sec:core}
% Core algorithm.
\begin{algorithm}[ht]
  % Declare IO markers.
  \SetKwInOut{Input}{Input}
  %
  \SetKwInOut{Output}{Output}

  % addcover: add a cover relation to a tree
  \SetKwFunction{addcover}{UpdCover}
  % freshnode: create a fresh node.
  \SetKwFunction{freshnode}{FreshNode}
  % initloc: the initial location of a program
  \SetKwFunction{initloc}{init}
  % uncoverleaves: collect the uncovered leaves in a tree
  \SetKwFunction{uncoveredleaves}{UncoveredLeaves}
  % unwind: unwind
  \SetKwFunction{unwind}{Unwind}

  % Declare sub-program markers (procedure).
  \SetKwProg{myproc}{Procedure}{}{}

  % Declare inputs:
  \Input{$\prognm$: a program; $\errloc$: an error location}
  %
  \Output{A decision as to whether $\errloc$ is unreachable in
    $\prognm$.}

  % Core-rec: case: frontier is empty
  \myproc{$\verifier$
    {$((\treenm, \locmapnm)$, $\emptyseq)$}}{
    \label{line:empty-frontier}
    \Return{$\true$} \label{line:return-safe}
  }

  % Core-rec: case: frontier is non-empty:
  \myproc{$\verifier'${$((\treenm, \locmapnm)$,
      %
      $\cons{\nodenm}{\frontiernm})$}}{
    \label{line:non-empty-frontier}
    % If n does not model the error location:
    \eIf{$\locmapnm(\nodenm) \not= \errloc$}{ \label{line:unwind-case}
      % Recurse on the updated tree and frontier.
      $\treenm' \assign \unwind(\treenm, \nodenm)$
      %
      \label{line:unwind-tree} \;
      % Update the frontier:
      $\frontiernm' \assign \frontiernm \union (\nodesnm_{\treenm'}
      \setminus \nodesnm_{\treenm})$ \label{line:unwind-frontier} \;
      % Recurse.
      \Return{$\verifier'(\treenm', \frontiernm')$}
      %
      \label{line:unwind-recurse} \;
      %
    }
    % If n is mapped to the error location:
    { \label{line:refine-case}
      % Case split: refinement of tree:
      \Switch{$\refine(\treenm, \errloc)$}{
        % Case: tree has no safe refinement.
        \lCase{$\unsafe$}{
          \label{line:bug-case}
          % then return that program is not safe.
          \Return{$\unsafe$} \label{line:bug-return}
        }
        % Case: tree has no safe refinement.
        \Case{$\treenm'$}{ \label{line:refinement-case}
          % update the tree
          $\treenm' \assign \addcover(\treenm)$
          %
          \label{line:refinement-tree} \;
          % update the frontier
          $\frontiernm' \assign \uncoveredleaves(\treenm')$
          %
          \label{line:refinement-frontier} \;
          %
          \Return{$\verifier'(\treenm', \frontiernm')$}
          %
          \label{line:refinement-return} \;
          %
        }
        %
      }
      %
    }
    %
  }
  % Allocate a fresh node.
  $\nodenm \assign \freshnode()$ \;
  % Construct the initial tree.
  $\treenm_0 \assign
  %
  ((\singleton{\nodenm}, \emptyset,
  %
  [\nodenm \mapsto \patuni], \emptyset),
  [ \nodenm \mapsto \initloc(\prognm) ])$ \label{line:tree-init} \;
  % Construct the initial frontier.
  $\frontiernm_0 \assign [ \nodenm ]$ \label{line:frontier-init} \;
  % Call recursive algorithm on tree with a single node.
  \Return{$\verifier'(\treenm_0, \frontiernm_0)$ }
  %
  \label{line:base-call} \;
  \caption{$\verifier$: the core verification algorithm.
    %
    $\verifier$ takes as input (1) a \lang program
    $\prognm$ and (2) an error control location $\errloc$, and
    determines if $\errloc$ is reachable in $\prognm$.}
\label{alg:core}
\end{algorithm}
% Give overview.
In this section, we describe the core \verifier algorithm, given in
\autoref{alg:core}.
%
$\verifier$ takes as input a program $\prognm$ and an error
control location $\errloc$, and attempts to determine if $\errloc$ is
unreachable in $\prognm$.
% Describe the recursive algorithm.
$\verifier$ runs a recursive procedure $\verifier'$, which maintains
an unwinding tree $\treenm$ of $\prognm$ and a \emph{frontier} of
uncovered tree nodes to analyze in order to extend $\treenm$ to be a
complete unwinding tree.
% Describe the base case: empty frontier.
If the set of frontier nodes is empty (\autoref{line:empty-frontier}),
then $\verifier$ determines that $\treenm$ is a complete
unwinding, and determines $\errloc$ is unreachable in $\prognm$
(\autoref{line:return-safe}).

% Give the inductive case of the recursive algorithm.
If the set of frontier nodes is non-empty, then $\verifier$
chooses a node $\nodenm \in \frontiernm$
(\autoref{line:non-empty-frontier}).
% Case: n is not at the error location.
If $\nodenm$ does not model the error location $\errloc$
(\autoref{line:unwind-case}), then $\verifier$ extends
$\treenm$ with a node that models each control successor of the
control location of $\nodenm$ to form a new unwinding tree $\treenm'$,
by invoking a procedure $\unwind$ (\autoref{line:unwind-tree}), adds
each new node in $\treenm_U$ to the remaining frontier nodes
$\frontiernm$ to construct a new frontier $\frontiernm'$
(\autoref{line:unwind-frontier}), and invokes itself recursively on
$\treenm_U$ and $\frontiernm'$ (\autoref{line:unwind-recurse}).

% Case: n is at the error location.
If $\nodenm$ models the error control location $\errloc$
(\autoref{line:refine-case}), then $\verifier$ invokes the procedure
$\refine$ (described in \autoref{sec:refine}) on $\treenm$ and
$\errloc$.
%
If $\refine$ returns the value $\unsafe$
(\autoref{line:bug-case}), then $\verifier$ determines
that $\errloc$ is reachable in $\prognm$ (\autoref{line:bug-return}).
%
Otherwise, if $\refine$ returns an unwinding tree $\treenm'$ whose
invariants refute all paths to $\nodenm$
(\autoref{line:refinement-case}), then $\verifier$ updates the cover
relation of $\treenm$ to form an unwinding tree $\treenm'$ by invoking
a procedure $\updcover$ (\autoref{line:refinement-tree}; $\updcover$
is described below),
%
collects the set of uncovered leaves $\frontiernm'$ in $\treenm'$ by
invoking a procedure $\uncoveredleaves$
(\autoref{line:refinement-frontier}), and invokes itself recursively
on $\treenm'$ and $\frontiernm'$ (\autoref{line:refinement-return}).

% Give the base application of the recursive algorithm.
$\verifier$ constructs an initial unwinding tree $\treenm_0$
consisting of a single node $\nodenm$ that models the initial control
location of $\prognm$ (\autoref{line:tree-init}), constructs an
initial frontier containing only $\nodenm$
(\autoref{line:frontier-init}), and invokes $\verifier'$ on
$\treenm_0$ and $\frontiernm_0$ (\autoref{line:base-call}).

% Give the function for updating the cover relation.
\paragraph{Design of $\updcover$}
%
The procedure $\updcover$ takes an unwinding tree $\treenm$ as input
and updates the cover relation of $\treenm$.
%
In particular, $\updcover$ performs a pre-order traversal of
$\treenm$.
%
At each node $\nodenm \in \nodesnm_{\treenm}$, $\updcover$ checks if
there is an uncovered non-descendant $\nodenm' \in \nodesnm_{\treenm}$
of $\nodenm$ such that (1) $\nodenm$ and $\nodenm'$ model the same
control location of $\prognm$ and (2) the invariant of $\nodenm$
entails the invariant of $\nodenm'$.
%
If so, $\updcover$ adds a cover edge from $\nodenm$ to $\nodenm'$ and
does not traverse any subtree of $\nodenm$.

% refine: give the refine operation.
\subsubsection{\refine}
\label{sec:refine}
% Present our implementation of getting trace invariants from users.
\begin{algorithm}[ht]
  % Declare data.

  % errheap:
  \SetKwData{errheap}{e}
  % errrestnm: the rest of the error leaves.
  \SetKwData{errrestnm}{errrest}

  % Declare functions
  \SetKwFunction{children}{Children}
  %
  \SetKwFunction{consunwind}{ConsUnwind}
  %
  \SetKwFunction{instrof}{Instr}
  %
  \SetKwFunction{leafrec}{ErrLeavesPat}
  %
  \SetKwFunction{leavesof}{Leaves}
  %
  \SetKwFunction{parentof}{Parent}
  %
  \SetKwFunction{refineroot}{RefineRoot}
  %
  \SetKwFunction{refinetree}{RefineTree}
  %
  \SetKwFunction{rootof}{Root}

  % Declare IO markers.
  \SetKwInOut{Input}{Input}
  %
  \SetKwInOut{Output}{Output}

  % Declare sub-program markers.
  \SetKwProg{myproc}{Procedure}{}{}

  % Inputs: unwinding tree, error location.
  \Input{$(\treenm, \locmapnm)$: an unwinding tree
    %
  }
  % Output: inductive invariants.
  \Output{$\treenm$, updated with safe inductive invariants.}
  % refine-tree: refine a tree rooted at a node.
  \myproc{\refinetree{$\treenm$, $\reachshape$}}{
    %
    $\rootnm \assign \rootof(\treenm)$ \label{line:get-root} \;
    % Get the shape for the current node.
    $\rootshape \assign \seplearner(T, \rootnm, \reachshape)$
    %
    \label{line:root-shape} \;
    % Refine each child of the root.
    \ForEach{$\childnm \in \children(r)$}{
      %
      \label{line:child-enum}
      % Cons shape for post-image.
      $\childshape \assign \abspost_D[\instrof(\rootnm, \childnm)]
      %
      (\rootshape)$
      \label{line:child-post} \;
      %
      $\refinedchild_c \assign
      %
      \refinetree(\restrict{\treenm}{\childnm}, \childshape)$
      %
      \label{line:child-refine} \;
      %
    }
    % Return the final tree constructed.
    \Return{$\consunwind(
      %
      \rootnm, \rootshape, \{ \refinedchild_c \}_{c})$}
    %
    \label{line:cons-unwind} \;
    %
  }
  % Refine the entire tree.
  \Return{$\refinetree(\treenm, \allheaps)$}
  %
  \label{line:refine-tree-base} \;
  %
  \caption{$\refine$: takes as input an unwinding tree $T$ and updates
    the invariants in $T$ to refute all paths to nodes that model the
    error location $\errloc$.}
  \label{alg:refine}
\end{algorithm}

% Describe the refine operation: interface through query.
$\refine$ takes as input an unwinding tree $\treenm$ and updates the
invariants of $\treenm$ to refute any path to a node that models the
error location $\errloc$.
% Define refinetree.
$\refine$ uses a procedure $\refinetree$
(\autoref{line:get-root}--\autoref{line:cons-unwind}), which takes as
input an unwinding tree $\treenm$ and a heap shape $\reachshape$ and
updates the invariants of $\treenm$ to be safe inductive invariants
such that $\reachshape$ entails the invariant of the root of the tree.
%
$\refinetree$ invokes a \emph{invariant learner} $\seplearner$ on $T$,
the root of $T$, and $\reachshape$ (\autoref{line:root-shape};
possible implementations of $\seplearner$ are described in
\autoref{sec:sep-learners}).
%
For each child $\childnm$ of the root of $T$
(\autoref{line:child-enum}), $\refinetree$ constructs the abstract
post-image of $S_n$ for the instruction connecting $r$ to $c$, denoted
$\childshape$ (\autoref{line:child-post}), and then invokes itself
recursively on $T$ restricted to $c$ and $\childshape$
(\autoref{line:child-refine}).
%
$\refinetree$ returns the unwinding tree constructed from the root
node, the shape $\rootshape$, and the refined children of $\rootnm$
(\autoref{line:cons-unwind}).

% Discuss how refine calls refinetree.
$\refine$ invokes $\refinetree$ on the input unwinding tree $\treenm$
and the universal shape $\allheaps$ for $D$
(\autoref{line:refine-tree-base}).

% Give separator-learning algorithms.
\subsection{Invariant learners}
\label{sec:sep-learners}
%
In this section, we define two procedures that can be used to
implement \seplearner, which is used in the definition of \refine.
%
We first define a partial characterization function
(\autoref{sec:tree-part-char}) and validator
(\autoref{sec:tree-validator}) from a fixed unwinding tree and node,
which we use in the definition of both procedures.
%
We then define the two procedures: one queries a shape-synthesizer
oracle (\autoref{sec:syn-sep-learner}) and the other queries a
shape-characterization oracle (\autoref{sec:syn-char-learner}).
%
We then compare practical issues of implementing each invariant
learner, in terms of requiring an end-user to serve as its
corresponding oracle (\autoref{sec:practical-oracles}).

% Define partial characterization function from a tree node
\subsubsection{A partial characterization for an unwinding tree}
\label{sec:tree-part-char}

\begin{algorithm}[ht]
  % Declare IO markers.
  \SetKwInOut{Input}{Input}
  %
  \SetKwInOut{Output}{Output}
  % Declare sub-program markers.
  \SetKwProg{myproc}{Procedure}{}{}
  % Inputs: unwinding tree, error location.
  \Input{A heap $h \in \heaps$.}
  % Output: inductive invariants.
  \Output{For fixed unwinding tree $T$, node $n \in \nodesnm_T$ with
    strongest valid invariant $S \in \shapes_D$, returns either (1) a
    Boolean value denoting a decision as to whether $h$ must be in the
    language of the invariant for $n$ or (2) the value $\unknown$.}
  % mustin: h must be in the separator if its in the post-image of the
  % separator for S_m.
  \lIf{$h \in S$}
  %
  {
    \Return{$\true$} \label{line:ret-true}
  }
  %
  \lIf{$\biglor \{ \post^{*}[n', e](h) \not= \undef
  | \locnm(e) = \errloc, n' \step^{*} e \}$}
  %
  {
    \Return{$\false$} \label{line:ret-false}
  }
  %
  \Return{$\unknown$} \label{line:ret-unknown}
  %
  \caption{$\treepartchar_{T, n, S}$: defines a partial
    characterization function for a fixed unwinding tree $T$, node $n
    \in \nodesnm_T$, and strongest valid invariant $S$ for $n$.}
  \label{alg:tree-part-char}
\end{algorithm}

% Walk through the partial characterization.
\autoref{alg:tree-part-char} contains pseudocode for an algorithm
$\treepartchar_{T, n, S}$ that, for fixed unwinding tree $T$ with node
$n \in \nodesnm_T$ and strongest valid invariant $S \in \shapes$ for
$n$, decides whether a given heap $h \in \heaps$ must be in the
invariant for $n$.
%
$\treepartchar_{T, n, S}$ checks if $h$ is in $S$, and if so, returns
$\true$ (\autoref{line:ret-true}).
%
$\treepartchar_{T, n, S}$ then checks if, for some node $e$ that
models the error location $\errloc$, $h$ transitions to some heap on
the sequence of instructions on the control path from $n$ to $e$, and
if so, returns $\false$ (\autoref{line:ret-false}).
%
Otherwise, $\treepartchar_{T, n, S}$ returns $\unknown$
(\autoref{line:ret-unknown}).

% Define a validator of shapes from an unwinding tree.
\subsubsection{A shape validator for an unwinding tree}
\label{sec:tree-validator}
%
\begin{algorithm}[ht]
  % Declare IO markers.
  \SetKwInOut{Input}{Input}
  %
  \SetKwInOut{Output}{Output}
  % Inputs: unwinding tree, error location.
  \Input{For shape domain $D$, a candidate shape $C \in \shapes_D$.
  }
  % Output:
  \Output{One of:
    %
    (1) a value $\isvalid$ denoting that $C$ is a valid invariant for
    $n$,
    %
    (2) $\true$ paired with a heap that must be accepted by any
    invariant for $n$ but is not accepted by $C$, or
    %
    (3) $\false$ paired with a heap that must not be accepted by any
    invariant for $n$ but is accepted by $C$.
    %
  }
  % Try to get a positive counter-model from the shape for m.
  \Switch{$\chksubsumes(S, C)$}{
    %
    \lCase{$h \in \heaps$:}
    %
    {\Return{$(\true, h)$}
      %
      \label{line:add-ind-pos}
      %
    }
    %
  }
  % Collect a set of negative counter-heaps.
  $\negcounters \assign \emptyset$ \label{line:init-negs} \;
  %
  \ForEach{$\{ e\ |\ \locnm(e) = \errloc, n \step^{*} e \}$
    %
    \label{line:enum-errs}}
  {
    % Case-split on the result of the subsumption check.
    \Switch{$\chksubsumes(\abspost^{*}[n', e](C), \noheaps_D)$}{
      \lCase{$h \in \heaps$:}
      %
      {\Return{$(\false, h)$}
        %
        \label{line:return-neg}
        %
      }
      %
    }
    %
  }
  \Return{$\isvalid$} \label{line:ret-valid} \;
  %
  \caption{$\treevalidate_{T, n, S}$: for a fixed unwinding tree $T$,
    control node $n \in \nodesnm_T$, and the strongest valid
    invariant $S$ for $n$, defines a validator for candidate shapes at
    $n$.
    %
  }
  \label{alg:validate-tree-shape}
\end{algorithm}

% Walk through the validator.
\autoref{alg:validate-tree-shape} contains pseudocode for a procedure
$\treevalidate_{T, n, S}$ that, for fixed unwinding tree $T$, node $n
\in \nodesnm_T$, and strongest valid invariant $S \in \shapes$ for
$n$, validates an input candidate invariant $C \in \shapes$.
% Check subsumption with strongest inductive shape.
$\treevalidate_{T, n, S}$ first checks that $S$ is subsumed by $C$
and, if not, returns a heap that is accepted by $S$ but is not
accepted by $C$ (\autoref{line:add-ind-pos}).
%
$\treevalidate_{T, n, S}$ then enumerates over the set of error nodes
that are reachable from $n$ in $T$ (\autoref{line:enum-errs}).
%
For each such error node $e$, if the abstract post-image of $C$ along
the control path from $n$ to $e$ does not subsume $\noheaps_D$, then
$\treevalidate_{T, n, S}$ returns a heap $h$ accepted by the
post-image (\autoref{line:return-neg}).
%
Otherwise, $\treevalidate_{T, n, S}$ returns that $C$ is a valid
invariant for $n$ (\autoref{line:ret-valid}).

% Define a separation learner that uses a synthesizer oracle.
\subsubsection{An invariant learner using a synthesizer oracle}
\label{sec:syn-sep-learner}
% Give an overview of the definition.
Given access to a shape-\emph{synthesizing} oracle $O$ for shape
domain $D$, we can learn an optimal invariant for $n$ by running $O$
on the partial characteristic function for the current unwinding tree
and $n$.

% Give the exact definition.
I.e., let $O \in \synoracles_D$ be a synthesizer oracle for domain
shape domain $D$.
%
For each unwinding tree $T$, node $n \in \nodesnm_T$, and strongest
valid invariant $S \in \shapes$ for $n$:
%
\[
\seplearner(T, n, S) = O(\treepartchar_{T, n, S})
\]

% Define a separation learner that uses a synthesizer oracle.
\subsubsection{An invariant learner using a characterization oracle}
\label{sec:syn-char-learner}
%
Given access to a shape-\emph{characterizing} oracle $O$ for an
automatically learnable shape domain $D$, we can learn an invariant for
node $n$ in unwinding tree $T$ by running a learning procedure $P$ for
$D$.
%
The \emph{total} characteristic function given to $P$ is implemented
by running $O$ with access to the \emph{partial} characteristic
function defined by $T$ and $n$ (\autoref{sec:tree-part-char}).
%
The \emph{validator} given to $P$ is implemented using the validator
defined by $T$ and $n$ (\autoref{sec:tree-validator}).

% Give the exact definition.
I.e., let $\shapelearner \in \shapelearners_D$ be a learning algorithm
for $D$ and let $O \in \charoracles_D$ by a characterization oracle
for $D$.
%
For each unwinding tree $T$ and node $n \in \nodesnm_T$ with strongest
valid invariant $S$:
%
\[
\seplearner(T, n, S) =
%
\shapelearner(O(\treepartchar_{T, n, S}), \treevalidate_{T, n, S})
\]

% Talk about practical issues in implementing each separator learner.
\subsubsection{Implementing invariant learners with end-users}
\label{sec:practical-oracles}
%
\paragraph{Validating the results of an oracles} Both synthesizing
oracles are characterizing oracles are defined as functions that
always generate (the characteristic functions of) solutions of given
optimal refinement inference problems.
%
In practice, an end user serving as a synthesizing oracle may generate
a shape that is not a valid refinement of a given partial
characteristic function, or may not be optimal.
%
We can extend the given implementations of invariant learners to
validate the shapes provided by a user by invoking $\treevalidate_{T,
  n, S}$, and rejecting shapes that are not valid refinements.
%
Extending the invariant learner to validate that a shape provided by
a user is optimal over all valid refinements depends on the optimality
ordering of the particular shape domain, and may not always be
feasible.

% Characteristic oracle.
We can extend the invariant learner that uses a characterizing oracle
to check if a given heap $h$ is defined by $\treepartchar$ before
passing $h$ to the user, precluding the possibility that the user will
provide an unsound answer, and saving effort for the user.

\paragraph{Advantages of using synthesizing oracles}
%
There are several potential advantages to developing an analysis that
uses a synthesizing oracle:
%
\begin{enumerate}
\item
  The shape domain used by the analysis does not necessarily need
  to be automatically learnable.
\item
  It may be the case for some analysis problems that it is easier for
  a user to write a single shape that describes their solution, rather
  than serving as a characteristic function for their solution.
  %
  In particular, this seems feasible if the analysis queries the user
  on the membership of large heaps.
\end{enumerate}

\paragraph{Advantages of using characterizing oracles}
%
Conversely, there are several potential advantages to developing an
analysis that uses a characterizing oracle:
%
\begin{enumerate}
\item
  A user serving as a characterizing oracle does not need to formally
  understand a shape domain: they only need to accept or reject
  example concrete shapes.
\item
  Responses to Boolean queries can be more easily aggregated and
  weighted than across multiple users: an analysis that uses multiple
  users to serve as an oracle can simply ask all users to decide
  membership of a heap, and take its final result to be the result of
  a vote among all users.
  %
  Combining multiple shapes provided by users as synthesizers may be
  feasible for some domains, but would likely be more complex.
\end{enumerate}

% Give the properties of the verifier.
\subsection{\verifier Properties}
\label{sec:properties}
%
In this section, we give the correctness properties of \verifier,
namely its soundness (\autoref{sec:soundness}) and relative
completeness (\autoref{sec:completeness}).
%
Proofs of all theorems will be given in an extended version of this
paper.

% Claim that proveit is sound.
\subsubsection{Soundness}
\label{sec:soundness}
%
\BH{soundness: verifier is sound, even when using unsound oracles}

% Claim that proveit is complete.
\subsubsection{Relative Completeness}
\label{sec:completeness}
% Overview of claim:
\BH{give result relating optimality ordering over shapes and
  invariants}

\subsubsection{Relativization of Proofs}
\label{sec:relativization}
\BH{Somesh: complete}

%%% Local Variables:
%%% mode: latex
%%% TeX-master: "p"
%%% End:

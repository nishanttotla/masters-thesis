% Background: give technical background
% \section{Background}
\label{ch:background}

In this section, we define the formalize requirements for lazy interpolation-based model checking, based on \cite{mcmillan06}. This applies to the standard domain of model checking for data programs, and we'll extend it to heap-manipulating programs later.

We will use standard first-order logic (FOL) and the notation $\mathcal{L}(\Sigma)$ to denote the set of well-formed formulas (\textit{wffs}) of FOL over a vocabulary $\Sigma$ of non-logical symbols. For a given formula or set of formulas $\phi$ we will use $\mathcal{L}(\phi)$ to denote \textit{wffs} over the vocabulary of $\phi$.

For every non-logical symbol $s$, we presume the existence of a unique symbol $s'$ (that is, $s$ with one prime added). We think of $s$ with $n$ primes added as representing the value of $s$ at $n$ time units in the future. For any formula or term $\phi$, we will use the notation $\phi^{\langle n \rangle}$ to denote the addition of $n$ primes to every symbol in $\phi$ (meaning $\phi$ at $n$ time units in the future). For any set $\Sigma$ of symbols, let $\Sigma'$ denote $\{ s' | s \in \Sigma \}$ and $\Sigma^{\langle n \rangle}$ denote $\{ s^{\langle n \rangle} | s \in \Sigma \}$.

\section{Modeling Programs}
\label{sec:modeling-programs}

We use FOL formulas to characterize programs. To this end, let $S$, the state vocabulary, be a set of individual variables and uninterpreted $n$-ary functional and propositional constants. A \textit{state formula} is a formula in $\mathcal{L}(S)$ (which may also include various interpreted symbols, such as $=$ and $+$). A \textit{transition formula} is a formula in $\mathcal{L}(S \cup S')$.

For our purposes, a \textit{program} is a tuple $(\Lambda, \Delta, l_i, l_f)$, where $\Lambda$ is a finite set of program locations, $\Delta$ is a set of $actions$, $l_i \in \Lambda$ is the initial location and $l_f \in \Lambda$ is the error location. An $action$ is a triple $(l, T, m)$, where $l,m \in \Lambda$ are respectively the entry and exit locations of the action, and $G$ is a transition formula. A $path$ $\pi$ of a program is a sequence of transitions of the form $(l_0, T_0, l_1)(l_1, T_1, l_2) \cdots (l_{n-1}, T_{n-1}, l_n)$. The path is an \textit{error path} when $l_0 = l_1$ and $l_n = l_f$. The unfolding $\mathcal{U}(\pi)$ of path $\pi$ is the sequence of formulas $T_0^{\langle 0 \rangle}, \cdots, T_{n-1}^{\langle n-1 \rangle}$, that is, the sequence of transition formulas $T_0, \cdots, T_{n-1}$, with each $T_i$ shifted $i$ time units into the future.

We will say that path $\pi$ is $feasible$ when $\bigwedge \mathcal{U}(\pi)$ is consistent. We can think of a model of $\bigwedge \mathcal{U}(\pi)$ as a concrete program execution, assigning a value ot every program variable at every time $0, \cdots, n-1$. A program is said to be $safe$ when every error path of the program is infeasible. An \textit{inductive invariant} of a program is a map $I : \Lambda \rightarrow \mathcal{L}(S)$, such that $I(l_i) \equiv \true$ and for every action $(l, T, m) \in \Delta$, $I(l) \wedge T$ implies $I(m)'$. A \textit{safety invariant} of a program is an inductive invariant such that $I(l_f) \equiv \false$. Existence of a safety invariant of a program implies that the program is safe.

To simplify the presentation of algorithms, we will assume that every location has at least one outgoing action. This can be made true without affecting program safety by adding self-loops.

\section{Interpolants from Proofs}
\label{sec:interpolants-from-proofs}

Given a pair of formulas $(A,B)$, such that $A \wedge B$ is inconsistent, an \textit{interpolant} for $(A,B)$ is a formula $\hat{A}$ with the following properties:

\begin{itemize}
  \item $A$ implies $\hat{A}$
  \item $\hat{A} \wedge B$ is unsatisfiable, and
  \item $\hat{A} \in \mathcal{L}(A) \cap \mathcal{L}(B)$
\end{itemize}

The Craig Interpolation lemma \cite{craig1957} states that an interpolant always exists for inconsistent formulas in FOL. To handle program paths, this idea can be generalized to sequences of formulas. That is, given a sequence of formulas $\Gamma = A_1, \cdots , A_n$, we say that $\hat{A_0},\cdots, \hat{A_n}$ is an \textit{interpolant} for $\Gamma$ when

\begin{itemize}
  \item $\hat{A_0} = \true$ and $\hat{A_n} = \false$ and,
  \item for all $1 \leq i \leq n, \hat{A_{i-1}} \wedge A_i$ implies $\hat{A_i}$ and
  \item for all $1 \leq i < n, \hat{A_i} \in (\mathcal{L}(A_1 \cdots A_i) \cap \mathcal{L}(A_{i+1}\cdots A_n))$
\end{itemize}

That is, the $i$-th element of the interpolant is a formula over the common vocabulary of the prefix $A_1 \cdots A_i$ and the suffix $A_{i+1} \cdots A_n$, and each interpolant implies the next, with $A_i$. If $\Gamma$ is quantifier-free, we can derive a quantifier-free interpolant for $\Gamma$ from a refutation of $\Gamma$, in certain interpreted theories \cite{mcmillan05}.

\section{Program Unwindings}

We now give a definition of a program unwinding, and describe an algorithm to construct a complete unwinding using interpolants. For two vertices $v$ and $w$ of a tree, we will write $w \sqsubset v$ when $w$ is a proper ancestor of $v$.

% define program unwinding
\begin{defn}
  \label{defn:prog-unwinding}
  An unwinding of a program $\mathcal{A} = (\Lambda, \Delta, l_i, l_f)$ is a quadruple $(V, E, M_v, M_e)$, where $(V, E)$ is a directed tree rooted at $\epsilon$, $M_v : V \rightarrow \Lambda$ is the vertex map, and $M_e : E \rightarrow \Delta$ is the edge map, such that:

  \begin{itemize}
    \item $M_v(\epsilon) = l_i$
    \item for every non-leaf vertex $v \in V$, for every action $(M_v(v), T, m) \in \Delta$, there exists an edge $(v,w) \in E$ such that $M_v(w) = m$ and $M_e(v,w) = T$
  \end{itemize}
\end{defn}

% define labeled program unwinding
\begin{defn}
  \label{defn:labeled-prog-unwinding}
  A labeled unwinding of a program $\mathcal{A} = (\Lambda, \Delta, l_i, l_f)$ is a triple $(U, \psi, \rhd)$, where

  \begin{itemize}
    \item $U = (V, E, M_v, M_e)$ is an unwinding of $\mathcal{A}$
    \item $\psi : V \rightarrow \mathcal{L}(S)$ is called the vertex labeling, and
    \item $\rhd \subseteq V \times V$ is called the covering relation
  \end{itemize}

  A vertex $v \in V$ is said to be covered iff there exists $(w,x) \in \rhd$ such that $w \sqsubseteq v$. The unwinding is said to be safe iff, for all $v \in V$, $M_v(v) = l_f$ implies $\psi(v) \equiv \false$. It is complete iff every leaf $v \in V$ is covered.
\end{defn}

% define well-labeled program unwinding
\begin{defn}
  \label{defn:well-labeled-prog-unwinding}
  A labeled unwinding $(U, \psi, \rhd)$ of a program $\mathcal{A} = (\Lambda, \Delta, l_i, l_f)$, where $U = (V, E, M_v, M_e)$, is said to be well-labeled iff:

  \begin{itemize}
    \item $\psi(\epsilon) \equiv \true$, and
    \item for every edge $(v,w) \in E$, $\psi(v) \wedge M_e(v,w)$ implies $\psi(w)'$, and
    \item for all $(v,w) \in \rhd$, $\psi(v) \Rightarrow \psi(w)$, and $w$ is not covered
  \end{itemize}

  A vertex $v \in V$ is said to be covered iff there exists $(w,x) \in \rhd$ such that $w \sqsubseteq v$. The unwinding is said to be safe iff, for all $v \in V$, $M_v(v) = l_f$ implies $\psi(v) \equiv \false$. It is complete iff every leaf $v \in V$ is covered.
\end{defn}

Notice that, if a vertex is covered, all its descendants are also covered. Moreover, we do not allow a covered vertex to cover another vertex.

\begin{thm}
  If there exists a safe, complete, well-labeled unwinding of program $\mathcal{A}$, then $\mathcal{A}$ is safe.
\end{thm}

This is Theorem 1 from \cite{mcmillan06}.

%%% Local Variables:
%%% mode: latex
%%% TeX-master: "p"
%%% End:

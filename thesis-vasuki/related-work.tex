\section{Related Work}
\label{sec:related-work}

\paragraph{Counter-example Guided Abstraction Refinement}
% CEGAR:
A broad class of verifiers of programs and transition systems have
been proposed that implement \emph{counterexample-example guided
  abstraction refinement (CEGAR)}~\cite{clarke03}.
%
The common structure of all of these analyses is that they maintain an
approximate model of the possible runs of a system, and refine the
model until it represents a proof of correctness by iteratively (1)
choosing a path of execution $p$ allowed by the model that, if
feasible, constitutes a property violation, (2) refuting the
feasibility of $p$, and (3) using the refutation to refine the paths
of execution allowed by the model.
% Relative Completeness of Abstraction Refinement for Software Model
% Checking: oracle is really angelic non-determinism.
Perhaps the CEGAR-based analysis that is most closely related to the
one proposed in this work is actually a \emph{theoretical} analysis
that chooses program facts from which to construct a refutation by
querying a \emph{widening oracle}~\cite{ball02}.
%
The key property of the oracle-guided analysis is that if there is
sequence of widenings that the oracle can possible choose to cause the
analysis to verify a program, then the analysis will eventually verify
the program successfully.
%
Because the oracle does not solve a distinct problem, but instead
provides values to the analysis, it can be viewed as an agent of
\emph{angelic non-determinism}~\cite{bodik10}.
% Us: 
While \verifier also queries an oracle, the oracle solves a problem
distinct from providing values to the analysis, namely an active
learning problem over both positive \emph{and negative} example
graphs.

-predicate abstraction: predicates in logic can't describe shapes
\cite{ball01,henzinger02,henzinger04}

-interpolation: also builds an unwinding tree of invariants, but still
can't describe
shapes\cite{albarghouthi12,heizmann10,mcmillan06,rummer13}

-active learning, Mahdu's paper: only works on bounded tuples of
lists~\cite{garg13}

\paragraph{Shape Analysis}

-memory graphs: TVLA~\cite{sagiv02}, SMG's\cite{dudka13}

-separation logic~\cite{calcagano11,reynolds02}

%%% Local Variables: 
%%% mode: latex
%%% TeX-master: "p"
%%% End: 
